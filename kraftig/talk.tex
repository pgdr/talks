\documentclass{beamer}
\newcommand{\megaskip}{\bigskip\bigskip\bigskip\bigskip}
\newcommand{\bigpause}{\bigskip\pause}
\begin{document}

\begin{frame}
  \frametitle{Talking about software development}

  We are \emph{software developers}, but talk surprisingly little about software
  development.

  \bigskip

  I want us to be more \emph{concerned with and actively talk} about software
  development.

  \megaskip\pause

  I notice that we don't do it like they do it.

  \bigskip

  So why care about \emph{our history}?
\end{frame}

\begin{frame}
  \frametitle{The Art of Unix Programming}

  \begin{quote}
    Those who do not understand Unix are condemned to reinvent it, poorly.
    ---~Henry Spencer
  \end{quote}

  \bigpause

  \begin{quote}
    This is the Unix philosophy: Write programs that do one thing and do it
    well.  Write programs to work together.  Write programs to handle text
    streams, because that is a universal interface. ---~Eric S Raymond
  \end{quote}

  \bigpause

  \begin{enumerate}
  \item Rule of Modularity: Write simple parts connected by clean interfaces.
  \item Rule of Clarity: Clarity is better than cleverness.
  \item Rule of Composition: Design programs to be connected to other programs.
  \item ...
  \end{enumerate}

  \textbf{Question}: How can we live up to these rules?

  % \begin{itemize}
  % \item (Design for simplicity; add complexity only where you must.)
  % \item (Write a big program only when it is clear by demonstration that nothing
  %   else will do.)
  % \item (In interface design, always do the least surprising thing.)
  % \item (When a program has nothing surprising to say, it should say nothing.)
  % \item (When you must fail, fail noisily and as soon as possible.)
  % \item (Programmer time is expensive; conserve it in preference to machine
  %   time.)
  % \end{itemize}
\end{frame}


\begin{frame}
  \frametitle{Version numbers}


Version numbering:

  \pause

\begin{itemize}
\item 25.3
\item 8.0.0960
\item 4.13.7
\item 4.13.b1
\item 4.0.2
\item 2.60.3
\item 3.0
\item 1.13.3
\item 1.13.0rc2
\item 0.20.3
\item 1.8.3.1
\end{itemize}

\pause

Version numbering:
\begin{itemize}
\item 2017.10
\item 2017.10
\end{itemize}
\end{frame}


\begin{frame}
  \frametitle{Software release cycle}

  Why the nagging about version numbering?

  \bigskip

  Well it's not about giving software a name consisting of numbers and dots!

  \megaskip

  \begin{quote}
    These are the facts of the case and they are undisputed.
  \end{quote}

  \megaskip

  (Skip if no intention of making reusable code)
\end{frame}


\begin{frame}
  \frametitle{The anatomy of the version number}

  Version numbering:  Major.Minor.Micro (Micro = Patch)

  \megaskip

  \begin{itemize}
  \item The major number should be increased whenever the API changes in an
    incompatible way.\pause
  \item The minor number should be increased whenever the API changes in a
    compatible way.\pause
  \item The micro number should be increased whenever the implementation
    changes, while the API does not.
  \end{itemize}

\end{frame}
\begin{frame}
  \frametitle{The anatomy of the version number, cont'd}

  Pre-alpha $\to$ alpha $\to$ beta $\to$ release candidate $\to$ gold

  \pause

  \begin{itemize}
  \item If Micro contains a letter, a=alpha, b=beta, rc=release candidate
  \item beta is intended stable, but may change
  \item rc is feature frozen
  \end{itemize}

  \pause

  Examples:
  \begin{itemize}
  \item 2.3.pre-alpha1
  \item 2.3.pre-alpha2
  \item 2.3.a1
  \item 2.3.a2
  \item 2.3.b1
  \item 2.3.rc1
  \item 2.3.rc2
  \end{itemize}
\end{frame}


\begin{frame}
  \frametitle{Why the obsession with version numbers?}

  Because better men than we paved the road.  They wrote Unix, GNU coreutils,
  Linux, all the software that we use and adore.  They found a way.

  \bigskip

  \begin{quote}
    The first and most important quality of modular code is encapsulation.
    % Well-encapsulated modules don't expose their internals to each other.  They
    % don't call into the middle of each others' implementations, and they don't
    % promiscuously share global data.

    They communicate using APIs---narrow, well-defined sets of procedure calls
    and data structures.  ---~Eric S. Raymond
  \end{quote}

  A version is defined by its API, its functionality

  \bigskip

  Once a function goes in, it must stay in until next major version!
\end{frame}

\begin{frame}
  \frametitle{Consequence of software development}

  Mantra: Bad code can be deleted, bad API is legacy

  % \begin{quote}
  %   it means that responsibility for interface usage errors belongs to the
  %   interface designer, not the interface user. ---~Scott Meyers
  % \end{quote}

  \begin{itemize}
  \item API = functionality
  \item code = machinery
  \end{itemize}

  Code is something that coincidentally makes the API work.

  \bigpause

  \begin{quote}
    Due to the required backwards compatibility there is certainly a
    code-complexity price related to this.  ---~Joakim
  \end{quote}
\end{frame}

\begin{frame}
  \begin{itemize}
  \item PEP-8 \pause
  \item YAGNI (You ain't gonna need it) \pause
  \item Test \emph{your} code, not others' \pause
  \item API: Simple things should be simple, complex things should be possible (\emph{interface usage errors belongs to the   interface designer ---~Scott Meyers})\pause
  \item Code is the enemy --- write less, delete, don't write \pause
  \item External facing APIs are where design up-front matters!
    \begin{itemize}
    \item Changing API is painful
    \item creating backwards incompatibility is horrible
    \item design carefully! (But keep simple things simple ...)
    \end{itemize} \pause
  \item If a function or method is more than 30LOC, break it up! \pause
  \item Refactor --- keep in mind that programming is about abstractions (and we
    discover new abstractions as we go along) \pause
  \item Always see your test fail once!  (Here's a question: Can we have a robot
    making random changes in code and see if tests fail?) \pause
  \item Continuously address technical debt.
  \end{itemize}
\end{frame}

\begin{frame}
  \frametitle{Code is mass}

  Good PR: +14, -521 --- Bad PR: +3123, -1

  \begin{quote}
    One of my most productive days was throwing away 1000 lines of code.
    ---~Ken Thompson
  \end{quote}

  Code is mass and has a weight.  And somebody is going to carry it.

  \bigpause

  If we have the choice between implementing a feature, and using an existing
  library, the pros and cons are:

  \begin{itemize}
  \item implement it yourself, you (or rather your team) carries the weight
  \item use somebody else's implementation, they carry the weight, you only
    carry the load of using that library (which may or may not be expensive)
  \end{itemize}
\end{frame}

\begin{frame}
  \frametitle{Code is mass cont'd}

  If all else is equal, \emph{less code} is better than \emph{more code}.  Fewer
  lines equals lower weight.  (Not invented here)

  \bigpause

  We should think in terms of code as being something that's just there for the
  API to work.

  \bigpause

  \begin{quote}
    Thou shalt study thy libraries and strive not to re-invent them without
    cause, that thy code may be short and readable and thy days pleasant and
    productive.

    ---~Henry Spencer's \emph{"The Ten Commandments for C Programmers"}
  \end{quote}

\end{frame}

\end{document}
