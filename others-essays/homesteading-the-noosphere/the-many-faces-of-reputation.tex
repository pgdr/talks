\section{The Many Faces of Reputation}


There are reasons general to every gift culture why peer repute (prestige) is
worth playing for:

First and most obviously, good reputation among one's peers is a primary reward.
We're wired to experience it that way for evolutionary reasons touched on
earlier.  (Many people learn to redirect their drive for prestige into various
sublimations that have no obvious connection to a visible peer group, such as
``honor'', ``ethical integrity'', ``piety'' etc.; this does not change the
underlying mechanism.)

Secondly, prestige is a good way (and in a pure gift economy, the only way) to
attract attention and cooperation from others.  If one is well known for
generosity, intelligence, fair dealing, leadership ability, or other good
qualities, it becomes much easier to persuade other people that they will gain
by association with you.

Thirdly, if your gift economy is in contact with or intertwined with an exchange
economy or a command hierarchy, your reputation may spill over and earn you
higher status there.

Beyond these general reasons, the peculiar conditions of the hacker culture make
prestige even more valuable than it would be in a `real world' gift culture.

The main `peculiar condition' is that the artifacts one gives away (or,
interpreted another way, are the visible sign of one's gift of energy and time)
are very complex.  Their value is nowhere near as obvious as that of material
gifts or exchange-economy money.  It is much harder to objectively distinguish a
fine gift from a poor one.  Accordingly, the success of a giver's bid for status
is delicately dependent on the critical judgement of peers.

Another peculiarity is the relative purity of the open-source culture.  Most
gift cultures are compromised—either by exchange-economy relationships such as
trade in luxury goods, or by command-economy relationships such as family or
clan groupings.  No significant analogues of these exist in the open-source
culture; thus, ways of gaining status other than by peer repute are virtually
absent.
