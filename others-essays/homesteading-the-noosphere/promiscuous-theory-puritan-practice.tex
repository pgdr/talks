\section{Promiscuous Theory, Puritan Practice}

Through all these changes, nevertheless, there remained a broad consensus theory
of what `free software' or `open source' is.  The clearest expression of this
common theory can be found in the various open-source licenses, all of which
have crucial common elements.

In 1997 these common elements were distilled into the Debian Free Software
Guidelines, which became the Open Source Definition.  Under the guidelines
defined by the OSD, an open-source license must protect an unconditional right
of any party to modify (and redistribute modified versions of) open-source
software.

Thus, the implicit theory of the OSD (and OSD-conformant licenses such as the
GPL, the BSD license, and Perl's Artistic License) is that anyone can hack
anything.  Nothing prevents half a dozen different people from taking any given
open-source product (such as, say the Free Software Foundations's gcc C
compiler), duplicating the sources, running off with them in different
evolutionary directions, but all claiming to be the product.

This kind of divergence is called a fork.  The most important characteristic of
a fork is that it spawns competing projects that cannot later exchange code,
splitting the potential developer community.  (There are phenomena that look
superficially like forking but are not, such as the proliferation of different
Linux distributions.  In these pseudo-forking cases there may be separate
projects, but they use mostly common code and can benefit from each other's
development efforts completely enough that they are neither technically nor
sociologically a waste, and are not perceived as forks.)

The open-source licenses do nothing to restrain forking, let alone
pseudo-forking; in fact, one could argue that they implicitly encourage both.
In practice, however, pseudo-forking is common but forking almost never happens.
Splits in major projects have been rare, and are always accompanied by
re-labeling and a large volume of public self-justification.  It is clear, in
such cases as the GNU Emacs/XEmacs split, or the gcc/egcs split, or the various
fissionings of the BSD splinter groups, that the splitters felt they were going
against a fairly powerful community norm [BSD].

In fact (and in contradiction to the anyone-can-hack-anything consensus theory)
the open-source culture has an elaborate but largely unadmitted set of ownership
customs.  These customs regulate who can modify software, the circumstances
under which it can be modified, and (especially) who has the right to
redistribute modified versions back to the community.

The taboos of a culture throw its norms into sharp relief.  Therefore, it will
be useful later on if we summarize some important ones here:
\begin{itemize}
\item There is strong social pressure against forking projects.  It does not
  happen except under plea of dire necessity, with much public
  self-justification, and requires a renaming.
\item Distributing changes to a project without the cooperation of the
  moderators is frowned upon, except in special cases like essentially trivial
  porting fixes.
\item Removing a person's name from a project history, credits, or maintainer
  list is absolutely not done without the person's explicit consent.
\end{itemize}

In the remainder of this essay, we shall examine these taboos and ownership
customs in detail.  We shall inquire not only into how they function but what
they reveal about the underlying social dynamics and incentive structures of the
open-source community.
