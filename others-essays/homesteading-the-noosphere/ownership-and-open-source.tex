\section{Ownership and Open Source}

What does `ownership' mean when property is infinitely reduplicable, highly
malleable, and the surrounding culture has neither coercive power relationships
nor material scarcity economics?

Actually, in the case of the open-source culture this is an easy question to
answer.  The owner of a software project is the person who has the exclusive
right, recognized by the community at large, to distribute modified versions.

(In discussing `ownership' in this section I will use the singular, as though
all projects are owned by some one person.  It should be understood, however,
that projects may be owned by groups.  We shall examine the internal dynamics of
such groups later on.)

According to the standard open-source licenses, all parties are equals in the
evolutionary game.  But in practice there is a very well-recognized distinction
between `official' patches, approved and integrated into the evolving software
by the publicly recognized maintainers, and `rogue' patches by third parties.
Rogue patches are unusual, and generally not trusted [RP].

That public redistribution is the fundamental issue is easy to establish.
Custom encourages people to patch software for personal use when necessary.
Custom is indifferent to people who redistribute modified versions within a
closed user or development group.  It is only when modifications are posted to
the open-source community in general, to compete with the original, that
ownership becomes an issue.

There are, in general, three ways to acquire ownership of an open-source
project.  One, the most obvious, is to found the project.  When a project has
had only one maintainer since its inception and the maintainer is still active,
custom does not even permit a question as to who owns the project.

The second way is to have ownership of the project handed to you by the previous
owner (this is sometimes known as `passing the baton').  It is well understood
in the community that project owners have a duty to pass projects to competent
successors when they are no longer willing or able to invest needed time in
development or maintenance work.

It is significant that in the case of major projects, such transfers of control
are generally announced with some fanfare.  While it is unheard of for the
open-source community at large to actually interfere in the owner's choice of
succession, customary practice clearly incorporates a premise that public
legitimacy is important.

For minor projects, it is generally sufficient for a change history included
with the project distribution to note the change of ownership.  The clear
presumption is that if the former owner has not in fact voluntarily transferred
control, he or she may reassert control with community backing by objecting
publicly within a reasonable period of time.

The third way to acquire ownership of a project is to observe that it needs work
and the owner has disappeared or lost interest.  If you want to do this, it is
your responsibility to make the effort to find the owner.  If you don't succeed,
then you may announce in a relevant place (such as a Usenet newsgroup dedicated
to the application area) that the project appears to be orphaned, and that you
are considering taking responsibility for it.

Custom demands that you allow some time to pass before following up with an
announcement that you have declared yourself the new owner.  In this interval,
if someone else announces that they have been actually working on the project,
their claim trumps yours.  It is considered good form to give public notice of
your intentions more than once.  You get more points for good form if you
announce in many relevant forums (related newsgroups, mailing lists), and still
more if you show patience in waiting for replies.  In general, the more visible
effort you make to allow the previous owner or other claimants to respond, the
better your claim if no response is forthcoming.

If you have gone through this process in sight of the project's user community,
and there are no objections, then you may claim ownership of the orphaned
project and so note in its history file.  This, however, is less secure than
being passed the baton, and you cannot expect to be considered fully legitimate
until you have made substantial improvements in the sight of the user community.

I have observed these customs in action for 20 years, going back to the pre-FSF
ancient history of open-source software.  They have several very interesting
features.  One of the most interesting is that most hackers have followed them
without being fully aware of doing so.  Indeed, this may be the first conscious
and reasonably complete summary ever to have been written down.

Another is that, for unconscious customs, they have been followed with
remarkable (even astonishing) consistency.  I have observed the evolution of
literally hundreds of open-source projects, and I can still count the number of
significant violations I have observed or heard about on my fingers.

Yet a third interesting feature is that as these customs have evolved over time,
they have done so in a consistent direction.  That direction has been to
encourage more public accountability, more public notice, and more care about
preserving the credits and change histories of projects in ways that (among
other things) establish the legitimacy of the present owners.

These features suggest that the customs are not accidental, but are products of
some kind of implicit agenda or generative pattern in the open-source culture
that is utterly fundamental to the way it operates.

An early respondent pointed out that contrasting the Internet hacker culture
with the cracker/pirate culture (the ``warez d00dz'' centered around
game-cracking and pirate bulletin-board systems) illuminates the generative
patterns of both rather well.  We'll return to the d00dz for contrast later in
this essay.
