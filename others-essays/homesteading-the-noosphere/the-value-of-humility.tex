\section{The Value of Humility}

Having established that prestige is central to the hacker culture's reward
mechanisms, we now need to understand why it has seemed so important that this
fact remain semi-covert and largely unadmitted.

The contrast with the pirate culture is instructive.  In that culture,
status-seeking behavior is overt and even blatant.  These crackers seek acclaim
for releasing ``zero-day warez'' (cracked software redistributed on the day of
the original uncracked version's release) but are closemouthed about how they do
it.  These magicians don't like to give away their tricks.  And, as a result,
the knowledge base of the cracker culture as a whole increases only slowly.

In the hacker community, by contrast, one's work is one's statement.  There's a
very strict meritocracy (the best craftsmanship wins) and there's a strong ethos
that quality should (indeed must) be left to speak for itself.  The best brag is
code that ``just works'', and that any competent programmer can see is good
stuff.  Thus, the hacker culture's knowledge base increases rapidly.

The taboo against ego-driven posturing therefore increases productivity.  But
that's a second-order effect; what is being directly protected here is the
quality of the information in the community's peer-evaluation system.  That is,
boasting or self-importance is suppressed because it behaves like noise tending
to corrupt the vital signals from experiments in creative and cooperative
behavior.

For very similar reasons, attacking the author rather than the code is not done.
There is an interesting subtlety here that reinforces the point; hackers feel
very free to flame each other over ideological and personal differences, but it
is unheard of for any hacker to publicly attack another's competence at
technical work (even private criticism is unusual and tends to be muted in
tone).  Bug-hunting and criticism are always project-labeled, not
person-labeled.

Furthermore, past bugs are not automatically held against a developer; the fact
that a bug has been fixed is generally considered more important than the fact
that one used to be there.  As one respondent observed, one can gain status by
fixing `Emacs bugs', but not by fixing `Richard Stallman's bugs'—and it would be
considered extremely bad form to criticize Stallman for old Emacs bugs that have
since been fixed.

This makes an interesting contrast with many parts of academia, in which
trashing putatively defective work by others is an important mode of gaining
reputation.  In the hacker culture, such behavior is rather heavily tabooed—so
heavily, in fact, that the absence of such behavior did not present itself to me
as a datum until that one respondent with an unusual perspective pointed it out
nearly a full year after this essay was first published!

The taboo against attacks on competence (not shared with academia) is even more
revealing than the (shared) taboo on posturing, because we can relate it to a
difference between academia and hackerdom in their communications and support
structures.

The hacker culture's medium of gifting is intangible, its communications
channels are poor at expressing emotional nuance, and face-to-face contact among
its members is the exception rather than the rule.  This gives it a lower
tolerance of noise than most other gift cultures, and goes a long way to explain
both the taboo against posturing and the taboo against attacks on competence.
Any significant incidence of flames over hackers' competence would intolerably
disrupt the culture's reputation scoreboard.

The same vulnerability to noise explains the model of public humility required
of the hacker community's tribal elders.  They must be seen to be free of boast
and posturing so the taboo against dangerous noise will hold.  [DC]

Talking softly is also functional if one aspires to be a maintainer of a
successful project; one must convince the community that one has good judgement,
because most of the maintainer's job is going to be judging other people's code.
Who would be inclined to contribute work to someone who clearly can't judge the
quality of their own code, or whose behavior suggests they will attempt to
unfairly hog the reputation return from the project? Potential contributors want
project leaders with enough humility and class to be able to to say, when
objectively appropriate, ``Yes, that does work better than my version, I'll use
it''—and to give credit where credit is due.

Yet another reason for humble behavior is that in the open source world, you
seldom want to give the impression that a project is `done'.  This might lead a
potential contributor not to feel needed.  The way to maximize your leverage is
to be humble about the state of the program.  If one does one's bragging through
the code, and then says ``Well shucks, it doesn't do x, y, and z, so it can't be
that good'', patches for x, y, and z will often swiftly follow.

Finally, I have personally observed that the self-deprecating behavior of some
leading hackers reflects a real (and not unjustified) fear of becoming the
object of a personality cult.  Linus Torvalds and Larry Wall both provide clear
and numerous examples of such avoidance behavior.  Once, on a dinner expedition
with Larry Wall, I joked ``You're the alpha hacker here—you get to pick the
restaurant''.  He flinched noticeably.  And rightly so; failing to distinguish
their shared values from the personalities of their leaders has ruined a good
many voluntary communities, a pattern of which Larry and Linus cannot fail to be
fully aware.  On the other hand, most hackers would love to have Larry's
problem, if they could but bring themselves to admit it.
