\section{Conclusion: From Custom to Customary Law}

We have examined the customs which regulate the ownership and control of
open-source software.  We have seen how they imply an underlying theory of
property rights homologous to the Lockean theory of land tenure.  We have
related that to an analysis of the hacker culture as a `gift culture' in which
participants compete for prestige by giving time, energy, and creativity away.
We have examined the implications of this analysis for conflict resolution in
the culture.

The next logical question to ask is "Why does this matter?" Hackers developed
these customs without conscious analysis and (up to now) have followed them
without conscious analysis.  It's not immediately clear that conscious analysis
has gained us anything practical—unless, perhaps, we can move from description
to prescription and deduce ways to improve the functioning of these customs.

We have found a close logical analogy for hacker customs in the theory of land
tenure under the Anglo-American common-law tradition.  Historically [Miller],
the European tribal cultures that invented this tradition improved their
dispute-resolution systems by moving from a system of unarticulated,
semi-conscious custom to a body of explicit customary law memorized by tribal
wisemen—and eventually, written down.

Perhaps, as our population rises and acculturation of all new members becomes
more difficult, it is time for the hacker culture to do something analogous—to
develop written codes of good practice for resolving the various sorts of
disputes that can arise in connection with open-source projects, and a tradition
of arbitration in which senior members of the community may be asked to mediate
disputes.

The analysis in this essay suggests the outlines of what such a code might look
like, making explicit that which was previously implicit.  No such codes could
be imposed from above; they would have to be voluntarily adopted by the founders
or owners of individual projects.  Nor could they be completely rigid, as the
pressures on the culture are likely to change over time.  Finally, for
enforcement of such codes to work, they would have to reflect a broad consensus
of the hacker tribe.

I have begun work on such a code, tentatively titled the "Malvern Protocol"
after the little town where I live.  If the general analysis in this paper
becomes sufficiently widely accepted, I will make the Malvern Protocol publicly
available as a model code for dispute resolution.  Parties interested in
critiquing and developing this code, or just offering feedback on whether they
think it's a good idea or not, are invited to contact me by email.
