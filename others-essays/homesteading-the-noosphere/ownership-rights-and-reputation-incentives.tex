\section{Ownership Rights and Reputation Incentives}

We are now in a position to pull together the previous analyses into a coherent
account of hacker ownership customs.  We understand the yield from homesteading
the noosphere now; it is peer repute in the gift culture of hackers, with all
the secondary gains and side effects that implies.

From this understanding, we can analyze the Lockean property customs of
hackerdom as a means of maximizing reputation incentives; of ensuring that peer
credit goes where it is due and does not go where it is not due.

The three taboos we observed above make perfect sense under this analysis.
One's reputation can suffer unfairly if someone else misappropriates or mangles
one's work; these taboos (and related customs) attempt to prevent this from
happening.  (Or, to put it more pragmatically, hackers generally refrain from
forking or rogue-patching others' projects in order to be able to deny
legitimacy to the same behavior practiced against themselves.)

\begin{itemize}
\item Forking projects is bad because it exposes pre-fork contributors to a
  reputation risk they can only control by being active in both child projects
  simultaneously after the fork.  (This would generally be too confusing or
  difficult to be practical.)
\item Distributing rogue patches (or, much worse, rogue binaries) exposes the
  owners to an unfair reputation risk.  Even if the official code is perfect,
  the owners will catch flak from bugs in the patches (but see [RP]).
\item Surreptitiously filing someone's name off a project is, in cultural
  context, one of the ultimate crimes.  Doing this steals the victim's gift to
  be presented as the thief's own.
\end{itemize}

Of course, forking a project or distributing rogue patches for it also directly
attacks the reputation of the original developer's group.  If I fork or
rogue-patch your project, I am saying: "you made a wrong decision by failing to
take the project where I am taking it"; and anyone who uses my forked variation
is endorsing this challenge.  But this in itself would be a fair challenge,
albeit extreme; it's the sharpest end of peer review.  It's therefore not
sufficient in itself to account for the taboos, though it doubtless contributes
force to them.

All three taboo behaviors inflict global harm on the open-source community as
well as local harm on the victim(s).  Implicitly they damage the entire
community by decreasing each potential contributor's perceived likelihood that
gift/productive behavior will be rewarded.

It's important to note that there are alternate candidate explanations for two
of these three taboos.

First, hackers often explain their antipathy to forking projects by bemoaning
the wasteful duplication of work it would imply as the child products evolve on
more-or-less parallel courses into the future.  They may also observe that
forking tends to split the co-developer community, leaving both child projects
with fewer brains to use than the parent.

A respondent has pointed out that it is unusual for more than one offspring of a
fork to survive with significant `market share' into the long term.  This
strengthens the incentives for all parties to cooperate and avoid forking,
because it's hard to know in advance who will be on the losing side and see a
lot of their work either disappear entirely or languish in obscurity.

It has also been pointed out that the simple fact that forks are likely to
produce contention and dispute is enough to motivate social pressure against
them.  Contention and dispute disrupt the teamwork that is necessary for each
individual contributor to reach his or her goals.

Dislike of rogue patches is often explained by the objection that they can
create compatibility problems between the daughter versions, complicate
bug-tracking enormously, and inflict work on maintainers who have quite enough
to do catching their own mistakes.

There is considerable truth to these explanations, and they certainly do their
bit to reinforce the Lockean logic of ownership.  But while intellectually
attractive, they fail to explain why so much emotion and territoriality gets
displayed on the infrequent occasions that the taboos get bent or broken—not
just by the injured parties, but by bystanders and observers who often react
quite harshly.  Cold-blooded concerns about duplication of work and maintainance
hassles simply do not sufficiently explain the observed behavior.

Then, too, there is the third taboo.  It's hard to see how anything but the
reputation-game analysis can explain this.  The fact that this taboo is seldom
analyzed much more deeply than ``It wouldn't be fair'' is revealing in its own
way, as we shall see in the next section.
