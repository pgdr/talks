\section{Acculturation Mechanisms and the Link to Academia}

An early version of this essay posed the following research question: how does
the community inform and instruct its members as to its customs? Are the customs
self-evident or self-organizing at a semi-conscious level? Are they taught by
example? Are they taught by explicit instruction?

Teaching by explicit instruction is clearly rare, if only because few explicit
descriptions of the culture's norms have existed for instructional use up to
now.

Many norms are taught by example.  To cite one very simple case, there is a norm
that every software distribution should have a file called README or READ.ME
that contains first-look instructions for browsing the distribution.  This
convention has been well established since at least the early 1980s; it has
even, occasionally, been written down.  But one normally derives it from looking
at many distributions.

On the other hand, some hacker customs are self-organizing once one has acquired
a basic (perhaps unconscious) understanding of the reputation game.  Most
hackers never have to be taught the three taboos I listed earlier in this essay,
or at least would claim if asked that they are self-evident rather than
transmitted.  This phenomenon invites closer analysis—and perhaps we can find
its explanation in the process by which hackers acquire knowledge about the
culture.

Many cultures use hidden clues (more precisely `mysteries' in the
religio/mystical sense) as an acculturation mechanism.  These are secrets that
are not revealed to outsiders, but are expected to be discovered or deduced by
the aspiring newbie.  To be accepted inside, one must demonstrate that one both
understands the mystery and has learned it in a culturally sanctioned way.

The hacker culture makes unusually conscious and extensive use of such clues or
tests.  We can see this process operating at at least three levels:
\begin{itemize}
\item Password-like specific mysteries.  As one example, there is a Usenet
  newsgroup called alt.sysadmin.recovery that has a very explicit such secret;
  you cannot post without knowing it, and knowing it is considered evidence you
  are fit to post.  The regulars have a strong taboo against revealing this
  secret.
\item The requirement of initiation into certain technical mysteries.  One must
  absorb a good deal of technical knowledge before one can give valued gifts
  (e.g.\ one must know at least one of the major computer languages).  This
  requirement functions in the large in the way hidden clues do in the small, as
  a filter for qualities (such as capability for abstract thinking, persistence,
  and mental flexibility) that are necessary to function in the culture.
\item Social-context mysteries.  One becomes involved in the culture through
  attaching oneself to specific projects.  Each project is a live social context
  of hackers that the would-be contributor has to investigate and understand
  socially as well as technically in order to function.  (Concretely, a common
  way one does this is by reading the project's web pages and/or email
  archives.) It is through these project groups that newbies experience the
  behavioral example of experienced hackers.
\end{itemize}

In the process of acquiring these mysteries, the would-be hacker picks up
contextual knowledge that (after a while) does make the three taboos and other
customs seem `self-evident'.

One might, incidentally, argue that the structure of the hacker gift culture
itself is its own central mystery.  One is not considered acculturated
(concretely: no one will call you a hacker) until one demonstrates a gut-level
understanding of the reputation game and its implied customs, taboos, and
usages.  But this is trivial; all cultures demand such understanding from
would-be joiners.  Furthermore the hacker culture evinces no desire to have its
internal logic and folkways kept secret—or, at least, nobody has ever flamed me
for revealing them!

Respondents to this essay too numerous to list have pointed out that hacker
ownership customs seem intimately related to (and may derive directly from) the
practices of the academic world, especially the scientific research commmunity.
This research community has similar problems in mining a territory of
potentially productive ideas, and exhibits very similar adaptive solutions to
those problems in the ways it uses peer review and reputation.

Since many hackers have had formative exposure to academia (it's common to learn
how to hack while in college) the extent to which academia shares adaptive
patterns with the hacker culture is of more than casual interest in
understanding how these customs are applied.

Obvious parallels with the hacker `gift culture' as I have characterized it
abound in academia.  Once a researcher achieves tenure, there is no need to
worry about survival issues.  (Indeed, the concept of tenure can probably be
traced back to an earlier gift culture in which ``natural philosophers'' were
primarily wealthy gentlemen with time on their hands to devote to research.) In
the absence of survival issues, reputation enhancement becomes the driving goal,
which encourages sharing of new ideas and research through journals and other
media.  This makes objective functional sense because scientific research, like
the hacker culture, relies heavily on the idea of `standing upon the shoulders
of giants', and not having to rediscover basic principles over and over again.

Some have gone so far as to suggest that hacker customs are merely a reflection
of the research community's folkways and have actually (in most cases) been
acquired there by individual hackers.  This probably overstates the case, if
only because hacker custom seems to be readily acquired by intelligent
high-schoolers!
