\section{The Joy of Hacking}


In making this `reputation game' analysis, by the way, I do not mean to devalue
or ignore the pure artistic satisfaction of designing beautiful software and
making it work.  Hackers all experience this kind of satisfaction and thrive on
it.  People for whom it is not a significant motivation never become hackers in
the first place, just as people who don't love music never become composers.

So perhaps we should consider another model of hacker behavior in which the pure
joy of craftsmanship is the primary motivation.  This `craftsmanship' model
would have to explain hacker custom as a way of maximizing both the
opportunities for craftsmanship and the quality of the results.  Does this
conflict with or suggest different results than the reputation game model?

Not really.  In examining the craftsmanship model, we come back to the same
problems that constrain hackerdom to operate like a gift culture.  How can one
maximize quality if there is no metric for quality? If scarcity economics
doesn't operate, what metrics are available besides peer evaluation? It appears
that any craftsmanship culture ultimately must structure itself through a
reputation game—and, in fact, we can observe exactly this dynamic in many
historical craftsmanship cultures from the medieval guilds onwards.

In one important respect, the craftsmanship model is weaker than the `gift
culture' model; by itself, it doesn't help explain the contradiction we began
this essay with.

Finally, the craftsmanship motivation itself may not be psychologically as far
removed from the reputation game as we might like to assume.  Imagine your
beautiful program locked up in a drawer and never used again.  Now imagine it
being used effectively and with pleasure by many people.  Which dream gives you
satisfaction?

Nevertheless, we'll keep an eye on the craftsmanship model.  It is intuitively
appealing to many hackers, and explains some aspects of individual behavior well
enough [HT].

After I published the first version of this essay on the Internet, an anonymous
respondent commented: ``You may not work to get reputation, but the reputation
is a real payment with consequences if you do the job well.'' This is a subtle
and important point.  The reputation incentives continue to operate whether or
not a craftsman is aware of them; thus, ultimately, whether or not a hacker
understands his own behavior as part of the reputation game, his behavior will
be shaped by that game.

Other respondents related peer-esteem rewards and the joy of hacking to the
levels above subsistence needs in Abraham Maslow's well-known `hierarchy of
values' model of human motivation [MH].  On this view, the joy of hacking
fulfills a self-actualization or transcendence need, which will not be
consistently expressed until lower-level needs (including those for physical
security and for `belongingness' or peer esteem) have been at least minimally
satisfied.  Thus, the reputation game may be critical in providing a social
context within which the joy of hacking can in fact become the individual's
primary motive.
