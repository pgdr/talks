\section{The Hacker Milieu as Gift Culture}


To understand the role of reputation in the open-source culture, it is helpful
to move from history further into anthropology and economics, and examine the
difference between exchange cultures and gift cultures.

Human beings have an innate drive to compete for social status; it's wired in by
our evolutionary history.  For the 90\% of hominid history that ran before the
invention of agriculture, our ancestors lived in small nomadic hunter-gatherer
bands.  High-status individuals (those most effective at informing coalitions
and persuading others to cooperate with them) got the healthiest mates and
access to the best food.  This drive for status expresses itself in different
ways, depending largely on the degree of scarcity of survival goods.

Most ways humans have of organizing are adaptations to scarcity and want.  Each
way carries with it different ways of gaining social status.

The simplest way is the command hierarchy.  In command hierarchies, scarce goods
are allocated by one central authority and backed up by force.  Command
hierarchies scale very poorly [Mal]; they become increasingly brutal and
inefficient as they get larger.  For this reason, command hierarchies above the
size of an extended family are almost always parasites on a larger economy of a
different type.  In command hierarchies, social status is primarily determined
by access to coercive power.

Our society is predominantly an exchange economy.  This is a sophisticated
adaptation to scarcity that, unlike the command model, scales quite well.
Allocation of scarce goods is done in a decentralized way through trade and
voluntary cooperation (and in fact, the dominating effect of competitive desire
is to produce cooperative behavior).  In an exchange economy, social status is
primarily determined by having control of things (not necessarily material
things) to use or trade.

Most people have implicit mental models for both of the above, and how they
interact with each other.  Government, the military, and organized crime (for
example) are command hierarchies parasitic on the broader exchange economy we
call `the free market'.  There's a third model, however, that is radically
different from either and not generally recognized except by anthropologists;
the gift culture.

Gift cultures are adaptations not to scarcity but to abundance.  They arise in
populations that do not have significant material-scarcity problems with
survival goods.  We can observe gift cultures in action among aboriginal
cultures living in ecozones with mild climates and abundant food.  We can also
observe them in certain strata of our own society, especially in show business
and among the very wealthy.

Abundance makes command relationships difficult to sustain and exchange
relationships an almost pointless game.  In gift cultures, social status is
determined not by what you control but by what you give away.

Thus the Kwakiutl chieftain's potlach party.  Thus the multi-millionaire's
elaborate and usually public acts of philanthropy.  And thus the hacker's long
hours of effort to produce high-quality open-source code.

For examined in this way, it is quite clear that the society of open-source
hackers is in fact a gift culture.  Within it, there is no serious shortage of
the `survival necessities'—disk space, network bandwidth, computing power.
Software is freely shared.  This abundance creates a situation in which the only
available measure of competitive success is reputation among one's peers.

This observation is not in itself entirely sufficient to explain the observed
features of hacker culture, however.  The crackers and warez d00dz have a gift
culture that thrives in the same (electronic) media as that of the hackers, but
their behavior is very different.  The group mentality in their culture is much
stronger and more exclusive than among hackers.  They hoard secrets rather than
sharing them; one is much more likely to find cracker groups distributing
sourceless executables that crack software than tips that give away how they did
it.  (For an inside perspective on this behavior, see [LW]).

What this shows, in case it wasn't obvious, is that there is more than one way
to run a gift culture.  History and values matter.  I have summarized the
history of the hacker culture in A Brief History of Hackerdom[HH]; the ways in
which it shaped present behavior are not mysterious.  Hackers have defined their
culture by a set of choices about the form that their competition will take.  It
is that form that we will examine in the remainder of this essay.
