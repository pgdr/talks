\section{Gift Outcompetes Exchange}

There is a more interesting possibility here.  I suspect academia and the hacker
culture share adaptive patterns not because they're genetically related, but
because they've both evolved the one most optimal social organization for what
they're trying to do, given the laws of nature and the instinctive wiring of
human beings.  The verdict of history seems to be that free-market capitalism is
the globally optimal way to cooperate for economic efficiency; perhaps, in a
similar way, the reputation-game gift culture is the globally optimal way to
cooperate for generating (and checking!) high-quality creative work.

Support for this theory becomes from a large body of psychological studies on
the interaction between art and reward [GNU].  These studies have received less
attention than they should, in part perhaps because their popularizers have
shown a tendency to overinterpret them into general attacks against the free
market and intellectual property.  Nevertheless, their results do suggest that
some kinds of scarcity-economics rewards actually decrease the productivity of
creative workers such as programmers.

Psychologist Theresa Amabile of Brandeis University, cautiously summarizing the
results of a 1984 study of motivation and reward, observed ``It may be that
commissioned work will, in general, be less creative than work that is done out
of pure interest.''.  Amabile goes on to observe that ``The more complex the
activity, the more it's hurt by extrinsic reward.'' Interestingly, the studies
suggest that flat salaries don't demotivate, but piecework rates and bonuses do.

Thus, it may be economically smart to give performance bonuses to people who
flip burgers or dug ditches, but it's probably smarter to decouple salary from
performance in a programming shop and let people choose their own projects (both
trends that the open-source world takes to their logical conclusions).  Indeed,
these results suggest that the only time it is a good idea to reward performance
in programming is when the programmer is so motivated that he or she would have
worked without the reward!

Other researchers in the field are willing to point a finger straight at the
issues of autonomy and creative control that so preoccupy hackers.  ``To the
extent one's experience of being self-determined is limited,'' said Richard
Ryan, associate psychology professor at the University of Rochester, ``one's
creativity will be reduced as well.''

In general, presenting any task as a means rather than an end in itself seems to
demotivate.  Even winning a competition with others or gaining peer esteem can
be demotivating in this way if the victory is experienced as work for reward
(which may explain why hackers are culturally prohibited from explicitly seeking
or claiming that esteem).

To complicate the management problem further, controlling verbal feedback seems
to be just as demotivating as piecework payment.  Ryan found that corporate
employees who were told, ``Good, you're doing as you should'' were
``significantly less intrinsically motivated than those who received feedback
informationally.''

It may still be intelligent to offer incentives, but they have to come without
attachments to avoid gumming up the works.  There is a critical difference (Ryan
observes) between saying, ``I'm giving you this reward because I recognize the
value of your work'', and ``You're getting this reward because you've lived up
to my standards.'' The first does not demotivate; the second does.

In these psychological observations we can ground a case that an open-source
development group will be substantially more productive (especially over the
long term, in which creativity becomes more critical as a productivity
multiplier) than an equivalently sized and skilled group of closed-source
programmers (de)motivated by scarcity rewards.

This suggests from a slightly different angle one of the speculations in The
Cathedral And The Bazaar; that, ultimately, the industrial/factory mode of
software production was doomed to be outcompeted from the moment capitalism
began to create enough of a wealth surplus that many programmers could live in a
post-scarcity gift culture.

Indeed, it seems the prescription for highest software productivity is almost a
Zen paradox; if you want the most efficient production, you must give up trying
to make programmers produce.  Handle their subsistence, give them their heads,
and forget about deadlines.  To a conventional manager this sounds crazily
indulgent and doomed—but it is exactly the recipe with which the open-source
culture is now clobbering its competition.
