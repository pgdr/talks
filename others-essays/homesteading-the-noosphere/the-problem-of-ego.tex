\section{The Problem of Ego}

At the beginning of this essay I mentioned that the unconscious adaptive
knowledge of a culture is often at odds with its conscious ideology.  We've seen
one major example of this already in the fact that Lockean ownership customs
have been widely followed despite the fact that they violate the stated intent
of the standard licenses.

I have observed another interesting example of this phenomenon when discussing
the reputation-game analysis with hackers.  This is that many hackers resisted
the analysis and showed a strong reluctance to admit that their behavior was
motivated by a desire for peer repute or, as I incautiously labeled it at the
time, `ego satisfaction'.

This illustrates an interesting point about the hacker culture.  It consciously
distrusts and despises egotism and ego-based motivations; self-promotion tends
to be mercilessly criticized, even when the community might appear to have
something to gain from it.  So much so, in fact, that the culture's `big men'
and tribal elders are required to talk softly and humorously deprecate
themselves at every turn in order to maintain their status.  How this attitude
meshes with an incentive structure that apparently runs almost entirely on ego
cries out for explanation.

A large part of it, certainly, stems from the generally negative Europo-American
attitude towards `ego'.  The cultural matrix of most hackers teaches them that
desiring ego satisfaction is a bad (or at least immature) motivation; that ego
is at best an eccentricity tolerable only in prima donnas and often an actual
sign of mental pathology.  Only sublimated and disguised forms like ``peer
repute'', ``self-esteem'', ``professionalism'' or ``pride of accomplishment''
are generally acceptable.

I could write an entire other essay on the unhealthy roots of this part of our
cultural inheritance, and the astonishing amount of self-deceptive harm we do by
believing (against all the evidence of psychology and behavior) that we ever
have truly `selfless' motives.  Perhaps I would, if Friedrich Wilhelm Nietzsche
and Ayn Rand had not already done an entirely competent job (whatever their
other failings) of deconstructing `altruism' into unacknowledged kinds of
self-interest.

But I am not doing moral philosophy or psychology here, so I will simply observe
one minor kind of harm done by the belief that ego is evil, which is this: it
has made it emotionally difficult for many hackers to consciously understand the
social dynamics of their own culture!

But we are not quite done with this line of investigation.  The surrounding
culture's taboo against visibly ego-driven behavior is so much intensified in
the hacker (sub)culture that one must suspect it of having some sort of special
adaptive function for hackers.  Certainly the taboo is weaker (or nonexistent)
among many other gift cultures, such as the peer cultures of theater people or
the very wealthy.
