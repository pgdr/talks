\section{Conflict and Conflict Resolution}

We've seen that within projects, an increasing complexity of roles is expressed
by a distribution of design authority and partial property rights.  While this
is an efficient way to distribute incentives, it also dilutes the authority of
the project leader—most importantly, it dilutes the leader's authority to squash
potential conflicts.

While technical arguments over design might seem the most obvious risk for
internecine conflict, they are seldom a serious cause of strife.  These are
usually relatively easily resolved by the territorial rule that authority
follows responsibility.

Another way of resolving conflicts is by seniority—if two contributors or groups
of contributors have a dispute, and the dispute cannot be resolved objectively,
and neither owns the territory of the dispute, the side that has put the most
work into the project as a whole (that is, the side with the most property
rights in the whole project) wins.

(Equivalently, the side with the least invested loses.  Interestingly this
happens to be the same heuristic that many relational database engines use to
resolve deadlocks.  When two threads are deadlocked over resources, the side
with the least invested in the current transaction is selected as the deadlock
victim and is terminated.  This usually selects the longest running transaction,
or the more senior, as the victor.)

These rules generally suffice to resolve most project disputes.  When they do
not, fiat of the project leader usually suffices.  Disputes that survive both
these filters are rare.

Conflicts do not, as a rule, become serious unless these two criteria
("authority follows responsibility" and "seniority wins") point in different
directions, and the authority of the project leader is weak or absent.  The most
obvious case in which this may occur is a succession dispute following the
disappearance of the project lead.  I have been in one fight of this kind.  It
was ugly, painful, protracted, only resolved when all parties became exhausted
enough to hand control to an outside person, and I devoutly hope I am never
anywhere near anything of the kind again.

Ultimately, all of these conflict-resolution mechanisms rest on the entire
hacker community's willingness to enforce them.  The only available enforcement
mechanisms are flaming and shunning—public condemnation of those who break
custom, and refusal to cooperate with them after they have done so.
