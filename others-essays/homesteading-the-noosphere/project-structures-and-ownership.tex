\section{Project Structures and Ownership}


The trivial case is that in which the project has a single owner/maintainer.  In
that case there is no possible conflict.  The owner makes all decisions and
collects all credit and blame.  The only possible conflicts are over succession
issues—who gets to be the new owner if the old one disappears or loses interest.
The community also has an interest, under issue (c), in preventing forking.
These interests are expressed by a cultural norm that an owner/maintainer should
publicly hand title to someone if he or she can no longer maintain the project.

The simplest non-trivial case is when a project has multiple co-maintainers
working under a single ``benevolent dictator'' who owns the project.  Custom
favors this mode for group projects; it has been shown to work on projects as
large as the Linux kernel or Emacs, and solves the ``who decides'' problem in a
way that is not obviously worse than any of the alternatives.

Typically, a benevolent-dictator organization evolves from an owner-maintainer
organization as the founder attracts contributors.  Even if the owner stays
dictator, it introduces a new level of possible disputes over who gets credited
for what parts of the project.

In this situation, custom places an obligation on the owner/dictator to credit
contributors fairly (through, for example, appropriate mentions in README or
history files).  In terms of the Lockean property model, this means that by
contributing to a project you earn part of its reputation return (positive or
negative).

Pursuing this logic, we see that a ``benevolent dictator'' does not in fact own
his entire project absolutely.  Though he has the right to make binding
decisions, he in effect trades away shares of the total reputation return in
exchange for others' work.  The analogy with sharecropping on a farm is almost
irresistible, except that a contributor's name stays in the credits and
continues to ``earn'' to some degree even after that contributor is no longer
active.

As benevolent-dictator projects add more participants, they tend to develop two
tiers of contributors; ordinary contributors and co-developers.  A typical path
to becoming a co-developer is taking responsibility for a major subsystem of the
project.  Another is to take the role of ``lord high fixer'', characterizing and
fixing many bugs.  In this way or others, co-developers are the contributors who
make a substantial and continuing investment of time in the project.

The subsystem-owner role is particularly important for our analysis and deserves
further examination.  Hackers like to say that ``authority follows
responsibility''.  A co-developer who accepts maintainance responsibility for a
given subsystem generally gets to control both the implementation of that
subsystem and its interfaces with the rest of the project, subject only to
correction by the project leader (acting as architect).  We observe that this
rule effectively creates enclosed properties on the Lockean model within a
project, and has exactly the same conflict-prevention role as other property
boundaries.

By custom, the ``dictator'' or project leader in a project with co-developers is
expected to consult with those co-developers on key decisions.  This is
especially so if the decision concerns a subsystem that a co-developer ``owns''
(that is, has invested time in and taken responsibility for).  A wise leader,
recognizing the function of the project's internal property boundaries, will not
lightly interfere with or reverse decisions made by subsystem owners.

Some very large projects discard the ``benevolent dictator'' model entirely.
One way to do this is turn the co-developers into a voting committee (as with
Apache).  Another is rotating dictatorship, in which control is occasionally
passed from one member to another within a circle of senior co-developers; the
Perl developers organize themselves this way.

Such complicated arrangements are widely considered unstable and difficult.
Clearly this perceived difficulty is largely a function of the known hazards of
design-by-committee, and of committees themselves; these are problems the hacker
culture consciously understands.  However, I think some of the visceral
discomfort hackers feel about committee or rotating-chair organizations is that
they're hard to fit into the unconscious Lockean model hackers use for reasoning
about the simpler cases.  It's problematic, in these complex organizations, to
do an accounting of either ownership in the sense of control or ownership of
reputation returns.  It's hard to see where the internal boundaries are, and
thus hard to avoid conflict unless the group enjoys an exceptionally high level
of harmony and trust.
