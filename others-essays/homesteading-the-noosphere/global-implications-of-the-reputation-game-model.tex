\section{Global Implications of the Reputation-Game Model}

The reputation-game analysis has some more implications that may not be
immediately obvious.  Many of these derive from the fact that one gains more
prestige from founding a successful project than from cooperating in an existing
one.  One also gains more from projects that are strikingly innovative, as
opposed to being `me, too' incremental improvements on software that already
exists.  On the other hand, software that nobody but the author understands or
has a need for is a non-starter in the reputation game, and it's often easier to
attract good notice by contributing to an existing project than it is to get
people to notice a new one.  Finally, it's much harder to compete with an
already successful project than it is to fill an empty niche.

Thus, there's an optimum distance from one's neighbors (the most similar
competing projects).  Too close and one's product will be a ``me, too!'' of
limited value, a poor gift (one would be better off contributing to an existing
project).  Too far away, and nobody will be able to use, understand, or perceive
the relevance of one's effort (again, a poor gift).  This creates a pattern of
homesteading in the noosphere that rather resembles that of settlers spreading
into a physical frontier—not random, but like a diffusion-limited fractal.
Projects tend to get started to fill functional gaps near the frontier (see [NO]
for further discussion of the lure of novelty).

Some very successful projects become `category killers'; nobody wants to
homestead anywhere near them because competing against the established base for
the attention of hackers would be too hard.  People who might otherwise found
their own distinct efforts end up, instead, adding extensions for these big,
successful projects.  The classic `category killer' example is GNU Emacs; its
variants fill the ecological niche for a fully-programmable editor so completely
that no competitor has gotten much beyond the one-man project stage since the
early 1980s.  Instead, people write Emacs modes.

Globally, these two tendencies (gap-filling and category-killers) have driven a
broadly predictable trend in project starts over time.  In the 1970s most of the
open source that existed was toys and demos.  In the 1980s the push was in
development and Internet tools.  In the 1990s the action shifted to operating
systems.  In each case, a new and more difficult level of problems was attacked
when the possibilities of the previous one had been nearly exhausted.

This trend has interesting implications for the near future.  In early 1998,
Linux looks very much like a category-killer for the niche `open-source
operating systems'—people who might otherwise write competing operating systems
are now writing Linux device drivers and extensions instead.  And most of the
lower-level tools the culture ever imagined having as open source already exist.
What's left?

Applications.  As the third millenium begins, it seems safe to predict that
open-source development effort will increasingly shift towards the last virgin
territory—programs for non-techies.  A clear early indicator was the development
of GIMP, the Photoshop-like image workshop that is open source's first major
application with the kind of end-user–friendly GUI interface considered de
rigueur in commercial applications for the last decade.  Another is the amount
of buzz surrounding application-toolkit projects like KDE and GNOME.

A respondent to this essay has pointed out that the homesteading analogy also
explains why hackers react with such visceral anger to Microsoft's ``embrace and
extend'' policy of complexifying and then closing up Internet protocols.  The
hacker culture can coexist with most closed software; the existence of Adobe
Photoshop, for example, does not make the territory near GIMP (its open-source
equivalent) significantly less attractive.  But when Microsoft succeeds at
de-commoditizing [HD] a protocol so that only Microsoft's own programmers can
write software for it, they do not merely harm customers by extending their
monopoly; they also reduce the amount and quality of noosphere available for
hackers to homestead and cultivate.  No wonder hackers often refer to
Microsoft's strategy as ``protocol pollution''; they are reacting exactly like
farmers watching someone poison the river they water their crops with!

Finally, the reputation-game analysis explains the oft-cited dictum that you do
not become a hacker by calling yourself a hacker—you become a hacker when other
hackers call you a hacker [KN].  A `hacker', considered in this light, is
somebody who has shown (by contributing gifts) that he or she both has technical
ability and understands how the reputation game works.  This judgement is mostly
one of awareness and acculturation, and can be delivered only by those already
well inside the culture.
