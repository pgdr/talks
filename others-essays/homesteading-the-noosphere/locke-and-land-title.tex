\section{Locke and Land Title}


To understand this generative pattern, it helps to notice a historical analogy
for these customs that is far outside the domain of hackers' usual concerns.  As
students of legal history and political philosophy may recognize, the theory of
property they imply is virtually identical to the Anglo-American common-law
theory of land tenure!

In this theory, there are three ways to acquire ownership of land:

On a frontier, where land exists that has never had an owner, one can acquire
ownership by homesteading, mixing one's labor with the unowned land, fencing it,
and defending one's title.

The usual means of transfer in settled areas is transfer of title—that is,
receiving the deed from the previous owner.  In this theory, the concept of
`chain of title' is important.  The ideal proof of ownership is a chain of deeds
and transfers extending back to when the land was originally homesteaded.

Finally, the common-law theory recognizes that land title may be lost or
abandoned (for example, if the owner dies without heirs, or the records needed
to establish chain of title to vacant land are gone).  A piece of land that has
become derelict in this way may be claimed by adverse possession—one moves in,
improves it, and defends title as if homesteading.

This theory, like hacker customs, evolved organically in a context where central
authority was weak or nonexistent.  It developed over a period of a thousand
years from Norse and Germanic tribal law.  Because it was systematized and
rationalized in the early modern era by the English political philosopher John
Locke, it is sometimes referred to as the Lockean theory of property.

Logically similar theories have tended to evolve wherever property has high
economic or survival value and no single authority is powerful enough to force
central allocation of scarce goods.  This is true even in the hunter-gatherer
cultures that are sometimes romantically thought to have no concept of
`property'.  For example, in the traditions of the !Kung San bushmen of the
Kgalagadi (formerly `Kalahari') Desert, there is no ownership of hunting
grounds.  But there is ownership of waterholes and springs under a theory
recognizably akin to Locke's.

The !Kung San example is instructive, because it shows that Lockean property
customs arise only where the expected return from the resource exceeds the
expected cost of defending it.  Hunting grounds are not property because the
return from hunting is highly unpredictable and variable, and (although highly
prized) not a necessity for day-to-day survival.  Waterholes, on the other hand,
are vital to survival and small enough to defend.

The `noosphere' of this essay's title is the territory of ideas, the space of
all possible thoughts [N].  What we see implied in hacker ownership customs is a
Lockean theory of property rights in one subset of the noosphere, the space of
all programs.  Hence `homesteading the noosphere', which is what every founder
of a new open-source project does.

Faré Rideau <fare@tunes.org> correctly points out that hackers do not exactly
operate in the territory of pure ideas.  He asserts that what hackers own is
programming projects—intensional focus points of material labor (development,
service, etc), to which are associated things like reputation, trustworthiness,
etc.  He therefore asserts that the space spanned by hacker projects, is not the
noosphere but a sort of dual of it, the space of noosphere-exploring program
projects.  (With an apologetic nod to the astrophysicists out there, it would be
etymologically correct to call this dual space the `ergosphere' or `sphere of
work'.)

In practice, the distinction between noosphere and ergosphere is not important
for the purposes of our present argument.  It is dubious whether the `noosphere'
in the pure sense on which Faré insists can be said to exist in any meaningful
way; one would almost have to be a Platonic philosopher to believe in it.  And
the distinction between noosphere and ergosphere is only of practical importance
if one wishes to assert that ideas (the elements of the noosphere) cannot be
owned, but their instantiations as projects can.  This question leads to issues
in the theory of intellectual property which are beyond the scope of this essay
(but see [DF]).

To avoid confusion, however, it is important to note that neither the noosphere
nor the ergosphere is the same as the totality of virtual locations in
electronic media that is sometimes (to the disgust of most hackers) called
`cyberspace'.  Property there is regulated by completely different rules that
are closer to those of the material substratum—essentially, he who owns the
media and machines on which a part of `cyberspace' is hosted owns that piece of
cyberspace as a result.

The Lockean logic of custom suggests strongly that open-source hackers observe
the customs they do in order to defend some kind of expected return from their
effort.  The return must be more significant than the effort of homesteading
projects, the cost of maintaining version histories that document `chain of
title', and the time cost of making public notifications and waiting before
taking adverse possession of an orphaned project.

Furthermore, the `yield' from open source must be something more than simply the
use of the software, something else that would be compromised or diluted by
forking.  If use were the only issue, there would be no taboo against forking,
and open-source ownership would not resemble land tenure at all.  In fact, this
alternate world (where use is the only yield, and forking is unproblematic) is
the one implied by existing open-source licenses.

We can eliminate some candidate kinds of yield right away.  Because you can't
coerce effectively over a network connection, seeking power is right out.
Likewise, the open-source culture doesn't have anything much resembling money or
an internal scarcity economy, so hackers cannot be pursuing anything very
closely analogous to material wealth (e.g.  the accumulation of scarcity
tokens).

There is one way that open-source activity can help people become wealthier,
however—a way that provides a valuable clue to what actually motivates it.
Occasionally, the reputation one gains in the hacker culture can spill over into
the real world in economically significant ways.  It can get you a better job
offer, or a consulting contract, or a book deal.

This kind of side effect, however, is at best rare and marginal for most
hackers; far too much so to make it convincing as a sole explanation, even if we
ignore the repeated protestations by hackers that they're doing what they do not
for money but out of idealism or love.

However, the way such economic side effects are mediated is worth examination.
Next we'll see that an understanding of the dynamics of reputation within the
open-source culture itself has considerable explanatory power.
