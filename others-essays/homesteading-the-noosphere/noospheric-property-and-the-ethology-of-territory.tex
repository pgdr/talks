\section{Noospheric Property and the Ethology of Territory}

To understand the causes and consequences of Lockean property customs, it will
help us to look at them from yet another angle; that of animal ethology,
specifically the ethology of territory.

Property is an abstraction of animal territoriality, which evolved as a way of
reducing intraspecies violence.  By marking his bounds, and respecting the
bounds of others, a wolf diminishes his chances of being in a fight that could
weaken or kill him and make him less reproductively successful.  Similarly, the
function of property in human societies is to prevent inter-human conflict by
setting bounds that clearly separate peaceful behavior from aggression.

It is fashionable in some circles to describe human property as an arbitrary
social convention, but this is dead wrong.  Anybody who has ever owned a dog who
barked when strangers came near its owner's property has experienced the
essential continuity between animal territoriality and human property.  Our
domesticated cousins of the wolf know, instinctively, that property is no mere
social convention or game, but a critically important evolved mechanism for the
avoidance of violence.  (This makes them smarter than a good many human
political theorists.)

Claiming property (like marking territory) is a performative act, a way of
declaring what boundaries will be defended.  Community support of property
claims is a way to minimize friction and maximize cooperative behavior.  These
things remain true even when the ``property claim'' is much more abstract than a
fence or a dog's bark, even when it's just the statement of the project
maintainer's name in a README file.  It's still an abstraction of
territoriality, and (like other forms of property) based in territorial
instincts evolved to assist conflict resolution.

This ethological analysis may at first seem very abstract and difficult to
relate to actual hacker behavior.  But it has some important consequences.  One
is in explaining the popularity of World Wide Web sites, and especially why
open-source projects with websites seem so much more `real' and substantial than
those without them.

Considered objectively, this seems hard to explain.  Compared to the effort
involved in originating and maintaining even a small program, a web page is
easy, so it's hard to consider a web page evidence of substance or unusual
effort.

Nor are the functional characteristics of the Web itself sufficient explanation.
The communication functions of a web page can be as well or better served by a
combination of an FTP site, a mailing list, and Usenet postings.  In fact it's
quite unusual for a project's routine communications to be done over the Web
rather than via a mailing list or newsgroup.  Why, then, the popularity of
websites as project homes?

The metaphor implicit in the term `home page' provides an important clue.  While
founding an open-source project is a territorial claim in the noosphere (and
customarily recognized as such) it is not a terribly compelling one on the
psychological level.  Software, after all, has no natural location and is
instantly reduplicable.  It's assimilable to our instinctive notions of
`territory' and `property', but only after some effort.

A project home page concretizes an abstract homesteading in the space of
possible programs by expressing it as `home' territory in the more
spatially-organized realm of the World Wide Web.  Descending from the noosphere
to `cyberspace' doesn't get us all the way to the real world of fences and
barking dogs yet, but it does hook the abstract property claim more securely to
our instinctive wiring about territory.  And this is why projects with web pages
seem more `real'.

This point is much strengthened by hyperlinks and the existence of good search
engines.  A project with a web page is much more likely to be noticed by
somebody exploring its neighborhood in the noosphere; others will link to it,
searches will find it.  A web page is therefore a better advertisement, a more
effective performative act, a stronger claim on territory.

This ethological analysis also encourages us to look more closely at mechanisms
for handling conflict in the open-source culture.  It leads us to expect that,
in addition to maximizing reputation incentives, ownership customs should also
have a role in preventing and resolving conflicts.
