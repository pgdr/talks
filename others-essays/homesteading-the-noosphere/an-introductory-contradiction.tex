\section{An Introductory Contradiction}

Anyone who watches the busy, tremendously productive world of Internet
open-source software for a while is bound to notice an interesting contradiction
between what open-source hackers say they believe and the way they actually
behave—between the official ideology of the open-source culture and its actual
practice.

Cultures are adaptive machines.  The open-source culture is a response to an
identifiable set of drives and pressures.  As usual, the culture's adaptation to
its circumstances manifests both as conscious ideology and as implicit,
unconscious or semi-conscious knowledge.  And, as is not uncommon, the
unconscious adaptations are partly at odds with the conscious ideology.

In this essay, I will dig around the roots of that contradiction, and use it to
discover those drives and pressures.  I will deduce some interesting things
about the hacker culture and its customs.  I will conclude by suggesting ways in
which the culture's implicit knowledge can be leveraged better.
