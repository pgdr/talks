\section{The Varieties of Hacker Ideology}

The ideology of the Internet open-source culture (what hackers say they believe)
is a fairly complex topic in itself.  All members agree that open source (that
is, software that is freely redistributable and can readily evolved and be
modified to fit changing needs) is a good thing and worthy of significant and
collective effort.  This agreement effectively defines membership in the
culture.  However, the reasons individuals and various subcultures give for this
belief vary considerably.

One degree of variation is zealotry; whether open source development is regarded
merely as a convenient means to an end (good tools and fun toys and an
interesting game to play) or as an end in itself.

A person of great zeal might say ``Free software is my life! I exist to create
useful, beautiful programs and information resources, and then give them away.''
A person of moderate zeal might say ``Open source is a good thing, which I am
willing to spend significant time helping happen''.  A person of little zeal
might say ``Yes, open source is okay sometimes.  I play with it and respect
people who build it''.

Another degree of variation is in hostility to commercial software and/or the
companies perceived to dominate the commercial software market.

A very anticommercial person might say ``Commercial software is theft and
hoarding.  I write free software to end this evil.'' A moderately anticommercial
person might say ``Commercial software in general is OK because programmers
deserve to get paid, but companies that coast on shoddy products and throw their
weight around are evil.'' An un-anticommercial person might say ``Commercial
software is okay, I just use and/or write open-source software because I like it
better''.  (Nowadays, given the growth of the open-source part of the industry
since the first public version of this essay, one might also hear ``Commercial
software is fine, as long as I get the source or it does what I want it to
do.'')

All nine of the attitudes implied by the cross-product of the categories
mentioned earlier are represented in the open-source culture.  It is worthwhile
to point out the distinctions because they imply different agendas, and
different adaptive and cooperative behaviors.

Historically, the most visible and best-organized part of the hacker culture has
been both very zealous and very anticommercial.  The Free Software Foundation
founded by Richard M.  Stallman (RMS) supported a great deal of open-source
development from the early 1980s forward, including tools like Emacs and GCC
which are still basic to the Internet open-source world, and seem likely to
remain so for the forseeable future.

For many years the FSF was the single most important focus of open-source
hacking, producing a huge number of tools still critical to the culture.  The
FSF was also long the only sponsor of open source with an institutional identity
visible to outside observers of the hacker culture.  They effectively defined
the term `free software', deliberately giving it a confrontational weight (which
the newer label `open source' just as deliberately avoids).

Thus, perceptions of the hacker culture from both within and without it tended
to identify the culture with the FSF's zealous attitude and perceived
anticommercial aims.  RMS himself denies he is anticommercial, but his program
has been so read by most people, including many of his most vocal partisans.
The FSF's vigorous and explicit drive to ``Stamp Out Software Hoarding!'' became
the closest thing to a hacker ideology, and RMS the closest thing to a leader of
the hacker culture.

The FSF's license terms, the ``General Public License'' (GPL), expresses the
FSF's attitudes.  It is very widely used in the open-source world.  North
Carolina's Metalab (formerly Sunsite) is the largest and most popular software
archive in the Linux world.  In July 1997 about half the Sunsite software
packages with explicit license terms used GPL.

But the FSF was never the only game in town.  There was always a quieter, less
confrontational and more market-friendly strain in the hacker culture.  The
pragmatists were loyal not so much to an ideology as to a group of engineering
traditions founded on early open-source efforts which predated the FSF.  These
traditions included, most importantly, the intertwined technical cultures of
Unix and the pre-commercial Internet.

The typical pragmatist attitude is only moderately anticommercial, and its major
grievance against the corporate world is not `hoarding' per se.  Rather it is
that world's perverse refusal to adopt superior approaches incorporating Unix
and open standards and open-source software.  If the pragmatist hates anything,
it is less likely to be `hoarders' in general than the current King Log of the
software establishment; formerly IBM, now Microsoft.

To pragmatists the GPL is important as a tool, rather than as an end in itself.
Its main value is not as a weapon against `hoarding', but as a tool for
encouraging software sharing and the growth of bazaar-modebazaar-mode
development communities.  The pragmatist values having good tools and toys more
than he dislikes commercialism, and may use high-quality commercial software
without ideological discomfort.  At the same time, his open-source experience
has taught him standards of technical quality that very little closed software
can meet.

For many years, the pragmatist point of view expressed itself within the hacker
culture mainly as a stubborn current of refusal to completely buy into the GPL
in particular or the FSF's agenda in general.  Through the 1980s and early
1990s, this attitude tended to be associated with fans of Berkeley Unix, users
of the BSD license, and the early efforts to build open-source Unixes from the
BSD source base.  These efforts, however, failed to build bazaar communities of
significant size, and became seriously fragmented and ineffective.

Not until the Linux explosion of early 1993–1994 did pragmatism find a real
power base.  Although Linus Torvalds never made a point of opposing RMS, he set
an example by looking benignly on the growth of a commercial Linux industry, by
publicly endorsing the use of high-quality commercial software for specific
tasks, and by gently deriding the more purist and fanatical elements in the
culture.

A side effect of the rapid growth of Linux was the induction of a large number
of new hackers for which Linux was their primary loyalty and the FSF's agenda
primarily of historical interest.  Though the newer wave of Linux hackers might
describe the system as ``the choice of a GNU generation'', most tended to
emulate Torvalds more than Stallman.

Increasingly it was the anticommercial purists who found themselves in a
minority.  How much things had changed would not become apparent until the
Netscape announcement in February 1998 that it would distribute Navigator 5.0 in
source.  This excited more interest in `free software' within the corporate
world.  The subsequent call to the hacker culture to exploit this unprecedented
opportunity and to re-label its product from `free software' to `open source'
was met with a level of instant approval that surprised everybody involved.

In a reinforcing development, the pragmatist part of the culture was itself
becoming polycentric by the mid-1990s.  Other semi-independent communities with
their own self-consciousness and charismatic leaders began to bud from the
Unix/Internet root stock.  Of these, the most important after Linux was the Perl
culture under Larry Wall.  Smaller, but still significant, were the traditions
building up around John Osterhout's Tcl and Guido van Rossum's Python languages.
All three of these communities expressed their ideological independence by
devising their own, non-GPL licensing schemes.
