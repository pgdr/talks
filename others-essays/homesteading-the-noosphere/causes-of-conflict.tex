\section{Causes of Conflict}

In conflicts over open-source software we can identify four major issues:
\begin{itemize}
\item Who gets to make binding decisions about a project?
\item Who gets credit or blame for what?
\item How to reduce duplication of effort and prevent rogue versions from
  complicating bug tracking?
\item What is the Right Thing, technically speaking?
\end{itemize}
If we take a second look at the ``What is the Right Thing'' issue, however, it
tends to vanish.  For any such question, either there is an objective way to
decide it accepted by all parties or there isn't.  If there is, game over and
everybody wins.  If there isn't, it reduces to ``Who decides?''.

Accordingly, the three problems a conflict-resolution theory has to resolve
about a project are (a) where the buck stops on design decisions, (b) how to
decide which contributors are credited and how, and (c) how to keep a project
group and product from fissioning into multiple branches.

The role of ownership customs in resolving issues (a) and (c) is clear.  Custom
affirms that the owners of the project make the binding decisions.  We have
previously observed that custom also exerts heavy pressure against dilution of
ownership by forking.

It's instructive to notice that these customs make sense even if one forgets the
reputation game and examines them from within a pure `craftmanship' model of the
hacker culture.  In this view these customs have less to do with the dilution of
reputation incentives than with protecting a craftsman's right to execute his
vision in his chosen way.

The craftsmanship model is not, however, sufficient to explain hacker customs
about issue (b), who gets credit for what—because a pure craftsman, one
unconcerned with the reputation game, would have no motive to care.  To analyze
these, we need to take the Lockean theory one step further and examine conflicts
and the operation of property rights within projects as well as between them.
