\documentclass{article}
\usepackage[utf8]{inputenc}
\usepackage{microtype}

\title{Homesteading the Noosphere}
\author{Eric Steven Raymond}
\date{1998--2000}
\begin{document}
\maketitle

\begin{abstract}
  After observing a contradiction between the official ideology defined by
  open-source licenses and the actual behavior of hackers, I examine the actual
  customs that regulate the ownership and control of open-source software.  I
  show that they imply an underlying theory of property rights homologous to the
  Lockean theory of land tenure.  I then relate that to an analysis of the
  hacker culture as a `gift culture' in which participants compete for prestige
  by giving time, energy, and creativity away.  Finally, I examine the
  consequences of this analysis for conflict resolution in the culture, and
  develop some prescriptive implications.
\end{abstract}


\section{An Introductory Contradiction}

Anyone who watches the busy, tremendously productive world of Internet
open-source software for a while is bound to notice an interesting contradiction
between what open-source hackers say they believe and the way they actually
behave—between the official ideology of the open-source culture and its actual
practice.

Cultures are adaptive machines.  The open-source culture is a response to an
identifiable set of drives and pressures.  As usual, the culture's adaptation to
its circumstances manifests both as conscious ideology and as implicit,
unconscious or semi-conscious knowledge.  And, as is not uncommon, the
unconscious adaptations are partly at odds with the conscious ideology.

In this essay, I will dig around the roots of that contradiction, and use it to
discover those drives and pressures.  I will deduce some interesting things
about the hacker culture and its customs.  I will conclude by suggesting ways in
which the culture's implicit knowledge can be leveraged better.

\section{The Varieties of Hacker Ideology}

The ideology of the Internet open-source culture (what hackers say they believe)
is a fairly complex topic in itself.  All members agree that open source (that
is, software that is freely redistributable and can readily evolved and be
modified to fit changing needs) is a good thing and worthy of significant and
collective effort.  This agreement effectively defines membership in the
culture.  However, the reasons individuals and various subcultures give for this
belief vary considerably.

One degree of variation is zealotry; whether open source development is regarded
merely as a convenient means to an end (good tools and fun toys and an
interesting game to play) or as an end in itself.

A person of great zeal might say ``Free software is my life! I exist to create
useful, beautiful programs and information resources, and then give them away.''
A person of moderate zeal might say ``Open source is a good thing, which I am
willing to spend significant time helping happen''.  A person of little zeal
might say ``Yes, open source is okay sometimes.  I play with it and respect
people who build it''.

Another degree of variation is in hostility to commercial software and/or the
companies perceived to dominate the commercial software market.

A very anticommercial person might say ``Commercial software is theft and
hoarding.  I write free software to end this evil.'' A moderately anticommercial
person might say ``Commercial software in general is OK because programmers
deserve to get paid, but companies that coast on shoddy products and throw their
weight around are evil.'' An un-anticommercial person might say ``Commercial
software is okay, I just use and/or write open-source software because I like it
better''.  (Nowadays, given the growth of the open-source part of the industry
since the first public version of this essay, one might also hear ``Commercial
software is fine, as long as I get the source or it does what I want it to
do.'')

All nine of the attitudes implied by the cross-product of the categories
mentioned earlier are represented in the open-source culture.  It is worthwhile
to point out the distinctions because they imply different agendas, and
different adaptive and cooperative behaviors.

Historically, the most visible and best-organized part of the hacker culture has
been both very zealous and very anticommercial.  The Free Software Foundation
founded by Richard M.  Stallman (RMS) supported a great deal of open-source
development from the early 1980s forward, including tools like Emacs and GCC
which are still basic to the Internet open-source world, and seem likely to
remain so for the forseeable future.

For many years the FSF was the single most important focus of open-source
hacking, producing a huge number of tools still critical to the culture.  The
FSF was also long the only sponsor of open source with an institutional identity
visible to outside observers of the hacker culture.  They effectively defined
the term `free software', deliberately giving it a confrontational weight (which
the newer label `open source' just as deliberately avoids).

Thus, perceptions of the hacker culture from both within and without it tended
to identify the culture with the FSF's zealous attitude and perceived
anticommercial aims.  RMS himself denies he is anticommercial, but his program
has been so read by most people, including many of his most vocal partisans.
The FSF's vigorous and explicit drive to ``Stamp Out Software Hoarding!'' became
the closest thing to a hacker ideology, and RMS the closest thing to a leader of
the hacker culture.

The FSF's license terms, the ``General Public License'' (GPL), expresses the
FSF's attitudes.  It is very widely used in the open-source world.  North
Carolina's Metalab (formerly Sunsite) is the largest and most popular software
archive in the Linux world.  In July 1997 about half the Sunsite software
packages with explicit license terms used GPL.

But the FSF was never the only game in town.  There was always a quieter, less
confrontational and more market-friendly strain in the hacker culture.  The
pragmatists were loyal not so much to an ideology as to a group of engineering
traditions founded on early open-source efforts which predated the FSF.  These
traditions included, most importantly, the intertwined technical cultures of
Unix and the pre-commercial Internet.

The typical pragmatist attitude is only moderately anticommercial, and its major
grievance against the corporate world is not `hoarding' per se.  Rather it is
that world's perverse refusal to adopt superior approaches incorporating Unix
and open standards and open-source software.  If the pragmatist hates anything,
it is less likely to be `hoarders' in general than the current King Log of the
software establishment; formerly IBM, now Microsoft.

To pragmatists the GPL is important as a tool, rather than as an end in itself.
Its main value is not as a weapon against `hoarding', but as a tool for
encouraging software sharing and the growth of bazaar-modebazaar-mode
development communities.  The pragmatist values having good tools and toys more
than he dislikes commercialism, and may use high-quality commercial software
without ideological discomfort.  At the same time, his open-source experience
has taught him standards of technical quality that very little closed software
can meet.

For many years, the pragmatist point of view expressed itself within the hacker
culture mainly as a stubborn current of refusal to completely buy into the GPL
in particular or the FSF's agenda in general.  Through the 1980s and early
1990s, this attitude tended to be associated with fans of Berkeley Unix, users
of the BSD license, and the early efforts to build open-source Unixes from the
BSD source base.  These efforts, however, failed to build bazaar communities of
significant size, and became seriously fragmented and ineffective.

Not until the Linux explosion of early 1993–1994 did pragmatism find a real
power base.  Although Linus Torvalds never made a point of opposing RMS, he set
an example by looking benignly on the growth of a commercial Linux industry, by
publicly endorsing the use of high-quality commercial software for specific
tasks, and by gently deriding the more purist and fanatical elements in the
culture.

A side effect of the rapid growth of Linux was the induction of a large number
of new hackers for which Linux was their primary loyalty and the FSF's agenda
primarily of historical interest.  Though the newer wave of Linux hackers might
describe the system as ``the choice of a GNU generation'', most tended to
emulate Torvalds more than Stallman.

Increasingly it was the anticommercial purists who found themselves in a
minority.  How much things had changed would not become apparent until the
Netscape announcement in February 1998 that it would distribute Navigator 5.0 in
source.  This excited more interest in `free software' within the corporate
world.  The subsequent call to the hacker culture to exploit this unprecedented
opportunity and to re-label its product from `free software' to `open source'
was met with a level of instant approval that surprised everybody involved.

In a reinforcing development, the pragmatist part of the culture was itself
becoming polycentric by the mid-1990s.  Other semi-independent communities with
their own self-consciousness and charismatic leaders began to bud from the
Unix/Internet root stock.  Of these, the most important after Linux was the Perl
culture under Larry Wall.  Smaller, but still significant, were the traditions
building up around John Osterhout's Tcl and Guido van Rossum's Python languages.
All three of these communities expressed their ideological independence by
devising their own, non-GPL licensing schemes.

\section{Promiscuous Theory, Puritan Practice}

Through all these changes, nevertheless, there remained a broad consensus theory
of what `free software' or `open source' is.  The clearest expression of this
common theory can be found in the various open-source licenses, all of which
have crucial common elements.

In 1997 these common elements were distilled into the Debian Free Software
Guidelines, which became the Open Source Definition.  Under the guidelines
defined by the OSD, an open-source license must protect an unconditional right
of any party to modify (and redistribute modified versions of) open-source
software.

Thus, the implicit theory of the OSD (and OSD-conformant licenses such as the
GPL, the BSD license, and Perl's Artistic License) is that anyone can hack
anything.  Nothing prevents half a dozen different people from taking any given
open-source product (such as, say the Free Software Foundations's gcc C
compiler), duplicating the sources, running off with them in different
evolutionary directions, but all claiming to be the product.

This kind of divergence is called a fork.  The most important characteristic of
a fork is that it spawns competing projects that cannot later exchange code,
splitting the potential developer community.  (There are phenomena that look
superficially like forking but are not, such as the proliferation of different
Linux distributions.  In these pseudo-forking cases there may be separate
projects, but they use mostly common code and can benefit from each other's
development efforts completely enough that they are neither technically nor
sociologically a waste, and are not perceived as forks.)

The open-source licenses do nothing to restrain forking, let alone
pseudo-forking; in fact, one could argue that they implicitly encourage both.
In practice, however, pseudo-forking is common but forking almost never happens.
Splits in major projects have been rare, and are always accompanied by
re-labeling and a large volume of public self-justification.  It is clear, in
such cases as the GNU Emacs/XEmacs split, or the gcc/egcs split, or the various
fissionings of the BSD splinter groups, that the splitters felt they were going
against a fairly powerful community norm [BSD].

In fact (and in contradiction to the anyone-can-hack-anything consensus theory)
the open-source culture has an elaborate but largely unadmitted set of ownership
customs.  These customs regulate who can modify software, the circumstances
under which it can be modified, and (especially) who has the right to
redistribute modified versions back to the community.

The taboos of a culture throw its norms into sharp relief.  Therefore, it will
be useful later on if we summarize some important ones here:
\begin{itemize}
\item There is strong social pressure against forking projects.  It does not
  happen except under plea of dire necessity, with much public
  self-justification, and requires a renaming.
\item Distributing changes to a project without the cooperation of the
  moderators is frowned upon, except in special cases like essentially trivial
  porting fixes.
\item Removing a person's name from a project history, credits, or maintainer
  list is absolutely not done without the person's explicit consent.
\end{itemize}

In the remainder of this essay, we shall examine these taboos and ownership
customs in detail.  We shall inquire not only into how they function but what
they reveal about the underlying social dynamics and incentive structures of the
open-source community.

\section{Ownership and Open Source}

What does `ownership' mean when property is infinitely reduplicable, highly
malleable, and the surrounding culture has neither coercive power relationships
nor material scarcity economics?

Actually, in the case of the open-source culture this is an easy question to
answer.  The owner of a software project is the person who has the exclusive
right, recognized by the community at large, to distribute modified versions.

(In discussing `ownership' in this section I will use the singular, as though
all projects are owned by some one person.  It should be understood, however,
that projects may be owned by groups.  We shall examine the internal dynamics of
such groups later on.)

According to the standard open-source licenses, all parties are equals in the
evolutionary game.  But in practice there is a very well-recognized distinction
between `official' patches, approved and integrated into the evolving software
by the publicly recognized maintainers, and `rogue' patches by third parties.
Rogue patches are unusual, and generally not trusted [RP].

That public redistribution is the fundamental issue is easy to establish.
Custom encourages people to patch software for personal use when necessary.
Custom is indifferent to people who redistribute modified versions within a
closed user or development group.  It is only when modifications are posted to
the open-source community in general, to compete with the original, that
ownership becomes an issue.

There are, in general, three ways to acquire ownership of an open-source
project.  One, the most obvious, is to found the project.  When a project has
had only one maintainer since its inception and the maintainer is still active,
custom does not even permit a question as to who owns the project.

The second way is to have ownership of the project handed to you by the previous
owner (this is sometimes known as `passing the baton').  It is well understood
in the community that project owners have a duty to pass projects to competent
successors when they are no longer willing or able to invest needed time in
development or maintenance work.

It is significant that in the case of major projects, such transfers of control
are generally announced with some fanfare.  While it is unheard of for the
open-source community at large to actually interfere in the owner's choice of
succession, customary practice clearly incorporates a premise that public
legitimacy is important.

For minor projects, it is generally sufficient for a change history included
with the project distribution to note the change of ownership.  The clear
presumption is that if the former owner has not in fact voluntarily transferred
control, he or she may reassert control with community backing by objecting
publicly within a reasonable period of time.

The third way to acquire ownership of a project is to observe that it needs work
and the owner has disappeared or lost interest.  If you want to do this, it is
your responsibility to make the effort to find the owner.  If you don't succeed,
then you may announce in a relevant place (such as a Usenet newsgroup dedicated
to the application area) that the project appears to be orphaned, and that you
are considering taking responsibility for it.

Custom demands that you allow some time to pass before following up with an
announcement that you have declared yourself the new owner.  In this interval,
if someone else announces that they have been actually working on the project,
their claim trumps yours.  It is considered good form to give public notice of
your intentions more than once.  You get more points for good form if you
announce in many relevant forums (related newsgroups, mailing lists), and still
more if you show patience in waiting for replies.  In general, the more visible
effort you make to allow the previous owner or other claimants to respond, the
better your claim if no response is forthcoming.

If you have gone through this process in sight of the project's user community,
and there are no objections, then you may claim ownership of the orphaned
project and so note in its history file.  This, however, is less secure than
being passed the baton, and you cannot expect to be considered fully legitimate
until you have made substantial improvements in the sight of the user community.

I have observed these customs in action for 20 years, going back to the pre-FSF
ancient history of open-source software.  They have several very interesting
features.  One of the most interesting is that most hackers have followed them
without being fully aware of doing so.  Indeed, this may be the first conscious
and reasonably complete summary ever to have been written down.

Another is that, for unconscious customs, they have been followed with
remarkable (even astonishing) consistency.  I have observed the evolution of
literally hundreds of open-source projects, and I can still count the number of
significant violations I have observed or heard about on my fingers.

Yet a third interesting feature is that as these customs have evolved over time,
they have done so in a consistent direction.  That direction has been to
encourage more public accountability, more public notice, and more care about
preserving the credits and change histories of projects in ways that (among
other things) establish the legitimacy of the present owners.

These features suggest that the customs are not accidental, but are products of
some kind of implicit agenda or generative pattern in the open-source culture
that is utterly fundamental to the way it operates.

An early respondent pointed out that contrasting the Internet hacker culture
with the cracker/pirate culture (the ``warez d00dz'' centered around
game-cracking and pirate bulletin-board systems) illuminates the generative
patterns of both rather well.  We'll return to the d00dz for contrast later in
this essay.

\section{Locke and Land Title}


To understand this generative pattern, it helps to notice a historical analogy
for these customs that is far outside the domain of hackers' usual concerns.  As
students of legal history and political philosophy may recognize, the theory of
property they imply is virtually identical to the Anglo-American common-law
theory of land tenure!

In this theory, there are three ways to acquire ownership of land:

On a frontier, where land exists that has never had an owner, one can acquire
ownership by homesteading, mixing one's labor with the unowned land, fencing it,
and defending one's title.

The usual means of transfer in settled areas is transfer of title—that is,
receiving the deed from the previous owner.  In this theory, the concept of
`chain of title' is important.  The ideal proof of ownership is a chain of deeds
and transfers extending back to when the land was originally homesteaded.

Finally, the common-law theory recognizes that land title may be lost or
abandoned (for example, if the owner dies without heirs, or the records needed
to establish chain of title to vacant land are gone).  A piece of land that has
become derelict in this way may be claimed by adverse possession—one moves in,
improves it, and defends title as if homesteading.

This theory, like hacker customs, evolved organically in a context where central
authority was weak or nonexistent.  It developed over a period of a thousand
years from Norse and Germanic tribal law.  Because it was systematized and
rationalized in the early modern era by the English political philosopher John
Locke, it is sometimes referred to as the Lockean theory of property.

Logically similar theories have tended to evolve wherever property has high
economic or survival value and no single authority is powerful enough to force
central allocation of scarce goods.  This is true even in the hunter-gatherer
cultures that are sometimes romantically thought to have no concept of
`property'.  For example, in the traditions of the !Kung San bushmen of the
Kgalagadi (formerly `Kalahari') Desert, there is no ownership of hunting
grounds.  But there is ownership of waterholes and springs under a theory
recognizably akin to Locke's.

The !Kung San example is instructive, because it shows that Lockean property
customs arise only where the expected return from the resource exceeds the
expected cost of defending it.  Hunting grounds are not property because the
return from hunting is highly unpredictable and variable, and (although highly
prized) not a necessity for day-to-day survival.  Waterholes, on the other hand,
are vital to survival and small enough to defend.

The `noosphere' of this essay's title is the territory of ideas, the space of
all possible thoughts [N].  What we see implied in hacker ownership customs is a
Lockean theory of property rights in one subset of the noosphere, the space of
all programs.  Hence `homesteading the noosphere', which is what every founder
of a new open-source project does.

Faré Rideau <fare@tunes.org> correctly points out that hackers do not exactly
operate in the territory of pure ideas.  He asserts that what hackers own is
programming projects—intensional focus points of material labor (development,
service, etc), to which are associated things like reputation, trustworthiness,
etc.  He therefore asserts that the space spanned by hacker projects, is not the
noosphere but a sort of dual of it, the space of noosphere-exploring program
projects.  (With an apologetic nod to the astrophysicists out there, it would be
etymologically correct to call this dual space the `ergosphere' or `sphere of
work'.)

In practice, the distinction between noosphere and ergosphere is not important
for the purposes of our present argument.  It is dubious whether the `noosphere'
in the pure sense on which Faré insists can be said to exist in any meaningful
way; one would almost have to be a Platonic philosopher to believe in it.  And
the distinction between noosphere and ergosphere is only of practical importance
if one wishes to assert that ideas (the elements of the noosphere) cannot be
owned, but their instantiations as projects can.  This question leads to issues
in the theory of intellectual property which are beyond the scope of this essay
(but see [DF]).

To avoid confusion, however, it is important to note that neither the noosphere
nor the ergosphere is the same as the totality of virtual locations in
electronic media that is sometimes (to the disgust of most hackers) called
`cyberspace'.  Property there is regulated by completely different rules that
are closer to those of the material substratum—essentially, he who owns the
media and machines on which a part of `cyberspace' is hosted owns that piece of
cyberspace as a result.

The Lockean logic of custom suggests strongly that open-source hackers observe
the customs they do in order to defend some kind of expected return from their
effort.  The return must be more significant than the effort of homesteading
projects, the cost of maintaining version histories that document `chain of
title', and the time cost of making public notifications and waiting before
taking adverse possession of an orphaned project.

Furthermore, the `yield' from open source must be something more than simply the
use of the software, something else that would be compromised or diluted by
forking.  If use were the only issue, there would be no taboo against forking,
and open-source ownership would not resemble land tenure at all.  In fact, this
alternate world (where use is the only yield, and forking is unproblematic) is
the one implied by existing open-source licenses.

We can eliminate some candidate kinds of yield right away.  Because you can't
coerce effectively over a network connection, seeking power is right out.
Likewise, the open-source culture doesn't have anything much resembling money or
an internal scarcity economy, so hackers cannot be pursuing anything very
closely analogous to material wealth (e.g.  the accumulation of scarcity
tokens).

There is one way that open-source activity can help people become wealthier,
however—a way that provides a valuable clue to what actually motivates it.
Occasionally, the reputation one gains in the hacker culture can spill over into
the real world in economically significant ways.  It can get you a better job
offer, or a consulting contract, or a book deal.

This kind of side effect, however, is at best rare and marginal for most
hackers; far too much so to make it convincing as a sole explanation, even if we
ignore the repeated protestations by hackers that they're doing what they do not
for money but out of idealism or love.

However, the way such economic side effects are mediated is worth examination.
Next we'll see that an understanding of the dynamics of reputation within the
open-source culture itself has considerable explanatory power.

\section{The Hacker Milieu as Gift Culture}


To understand the role of reputation in the open-source culture, it is helpful
to move from history further into anthropology and economics, and examine the
difference between exchange cultures and gift cultures.

Human beings have an innate drive to compete for social status; it's wired in by
our evolutionary history.  For the 90\% of hominid history that ran before the
invention of agriculture, our ancestors lived in small nomadic hunter-gatherer
bands.  High-status individuals (those most effective at informing coalitions
and persuading others to cooperate with them) got the healthiest mates and
access to the best food.  This drive for status expresses itself in different
ways, depending largely on the degree of scarcity of survival goods.

Most ways humans have of organizing are adaptations to scarcity and want.  Each
way carries with it different ways of gaining social status.

The simplest way is the command hierarchy.  In command hierarchies, scarce goods
are allocated by one central authority and backed up by force.  Command
hierarchies scale very poorly [Mal]; they become increasingly brutal and
inefficient as they get larger.  For this reason, command hierarchies above the
size of an extended family are almost always parasites on a larger economy of a
different type.  In command hierarchies, social status is primarily determined
by access to coercive power.

Our society is predominantly an exchange economy.  This is a sophisticated
adaptation to scarcity that, unlike the command model, scales quite well.
Allocation of scarce goods is done in a decentralized way through trade and
voluntary cooperation (and in fact, the dominating effect of competitive desire
is to produce cooperative behavior).  In an exchange economy, social status is
primarily determined by having control of things (not necessarily material
things) to use or trade.

Most people have implicit mental models for both of the above, and how they
interact with each other.  Government, the military, and organized crime (for
example) are command hierarchies parasitic on the broader exchange economy we
call `the free market'.  There's a third model, however, that is radically
different from either and not generally recognized except by anthropologists;
the gift culture.

Gift cultures are adaptations not to scarcity but to abundance.  They arise in
populations that do not have significant material-scarcity problems with
survival goods.  We can observe gift cultures in action among aboriginal
cultures living in ecozones with mild climates and abundant food.  We can also
observe them in certain strata of our own society, especially in show business
and among the very wealthy.

Abundance makes command relationships difficult to sustain and exchange
relationships an almost pointless game.  In gift cultures, social status is
determined not by what you control but by what you give away.

Thus the Kwakiutl chieftain's potlach party.  Thus the multi-millionaire's
elaborate and usually public acts of philanthropy.  And thus the hacker's long
hours of effort to produce high-quality open-source code.

For examined in this way, it is quite clear that the society of open-source
hackers is in fact a gift culture.  Within it, there is no serious shortage of
the `survival necessities'—disk space, network bandwidth, computing power.
Software is freely shared.  This abundance creates a situation in which the only
available measure of competitive success is reputation among one's peers.

This observation is not in itself entirely sufficient to explain the observed
features of hacker culture, however.  The crackers and warez d00dz have a gift
culture that thrives in the same (electronic) media as that of the hackers, but
their behavior is very different.  The group mentality in their culture is much
stronger and more exclusive than among hackers.  They hoard secrets rather than
sharing them; one is much more likely to find cracker groups distributing
sourceless executables that crack software than tips that give away how they did
it.  (For an inside perspective on this behavior, see [LW]).

What this shows, in case it wasn't obvious, is that there is more than one way
to run a gift culture.  History and values matter.  I have summarized the
history of the hacker culture in A Brief History of Hackerdom[HH]; the ways in
which it shaped present behavior are not mysterious.  Hackers have defined their
culture by a set of choices about the form that their competition will take.  It
is that form that we will examine in the remainder of this essay.

\section{The Joy of Hacking}


In making this `reputation game' analysis, by the way, I do not mean to devalue
or ignore the pure artistic satisfaction of designing beautiful software and
making it work.  Hackers all experience this kind of satisfaction and thrive on
it.  People for whom it is not a significant motivation never become hackers in
the first place, just as people who don't love music never become composers.

So perhaps we should consider another model of hacker behavior in which the pure
joy of craftsmanship is the primary motivation.  This `craftsmanship' model
would have to explain hacker custom as a way of maximizing both the
opportunities for craftsmanship and the quality of the results.  Does this
conflict with or suggest different results than the reputation game model?

Not really.  In examining the craftsmanship model, we come back to the same
problems that constrain hackerdom to operate like a gift culture.  How can one
maximize quality if there is no metric for quality? If scarcity economics
doesn't operate, what metrics are available besides peer evaluation? It appears
that any craftsmanship culture ultimately must structure itself through a
reputation game—and, in fact, we can observe exactly this dynamic in many
historical craftsmanship cultures from the medieval guilds onwards.

In one important respect, the craftsmanship model is weaker than the `gift
culture' model; by itself, it doesn't help explain the contradiction we began
this essay with.

Finally, the craftsmanship motivation itself may not be psychologically as far
removed from the reputation game as we might like to assume.  Imagine your
beautiful program locked up in a drawer and never used again.  Now imagine it
being used effectively and with pleasure by many people.  Which dream gives you
satisfaction?

Nevertheless, we'll keep an eye on the craftsmanship model.  It is intuitively
appealing to many hackers, and explains some aspects of individual behavior well
enough [HT].

After I published the first version of this essay on the Internet, an anonymous
respondent commented: ``You may not work to get reputation, but the reputation
is a real payment with consequences if you do the job well.'' This is a subtle
and important point.  The reputation incentives continue to operate whether or
not a craftsman is aware of them; thus, ultimately, whether or not a hacker
understands his own behavior as part of the reputation game, his behavior will
be shaped by that game.

Other respondents related peer-esteem rewards and the joy of hacking to the
levels above subsistence needs in Abraham Maslow's well-known `hierarchy of
values' model of human motivation [MH].  On this view, the joy of hacking
fulfills a self-actualization or transcendence need, which will not be
consistently expressed until lower-level needs (including those for physical
security and for `belongingness' or peer esteem) have been at least minimally
satisfied.  Thus, the reputation game may be critical in providing a social
context within which the joy of hacking can in fact become the individual's
primary motive.

\section{The Many Faces of Reputation}


There are reasons general to every gift culture why peer repute (prestige) is
worth playing for:

First and most obviously, good reputation among one's peers is a primary reward.
We're wired to experience it that way for evolutionary reasons touched on
earlier.  (Many people learn to redirect their drive for prestige into various
sublimations that have no obvious connection to a visible peer group, such as
``honor'', ``ethical integrity'', ``piety'' etc.; this does not change the
underlying mechanism.)

Secondly, prestige is a good way (and in a pure gift economy, the only way) to
attract attention and cooperation from others.  If one is well known for
generosity, intelligence, fair dealing, leadership ability, or other good
qualities, it becomes much easier to persuade other people that they will gain
by association with you.

Thirdly, if your gift economy is in contact with or intertwined with an exchange
economy or a command hierarchy, your reputation may spill over and earn you
higher status there.

Beyond these general reasons, the peculiar conditions of the hacker culture make
prestige even more valuable than it would be in a `real world' gift culture.

The main `peculiar condition' is that the artifacts one gives away (or,
interpreted another way, are the visible sign of one's gift of energy and time)
are very complex.  Their value is nowhere near as obvious as that of material
gifts or exchange-economy money.  It is much harder to objectively distinguish a
fine gift from a poor one.  Accordingly, the success of a giver's bid for status
is delicately dependent on the critical judgement of peers.

Another peculiarity is the relative purity of the open-source culture.  Most
gift cultures are compromised—either by exchange-economy relationships such as
trade in luxury goods, or by command-economy relationships such as family or
clan groupings.  No significant analogues of these exist in the open-source
culture; thus, ways of gaining status other than by peer repute are virtually
absent.

\section{Ownership Rights and Reputation Incentives}

We are now in a position to pull together the previous analyses into a coherent
account of hacker ownership customs.  We understand the yield from homesteading
the noosphere now; it is peer repute in the gift culture of hackers, with all
the secondary gains and side effects that implies.

From this understanding, we can analyze the Lockean property customs of
hackerdom as a means of maximizing reputation incentives; of ensuring that peer
credit goes where it is due and does not go where it is not due.

The three taboos we observed above make perfect sense under this analysis.
One's reputation can suffer unfairly if someone else misappropriates or mangles
one's work; these taboos (and related customs) attempt to prevent this from
happening.  (Or, to put it more pragmatically, hackers generally refrain from
forking or rogue-patching others' projects in order to be able to deny
legitimacy to the same behavior practiced against themselves.)

\begin{itemize}
\item Forking projects is bad because it exposes pre-fork contributors to a
  reputation risk they can only control by being active in both child projects
  simultaneously after the fork.  (This would generally be too confusing or
  difficult to be practical.)
\item Distributing rogue patches (or, much worse, rogue binaries) exposes the
  owners to an unfair reputation risk.  Even if the official code is perfect,
  the owners will catch flak from bugs in the patches (but see [RP]).
\item Surreptitiously filing someone's name off a project is, in cultural
  context, one of the ultimate crimes.  Doing this steals the victim's gift to
  be presented as the thief's own.
\end{itemize}

Of course, forking a project or distributing rogue patches for it also directly
attacks the reputation of the original developer's group.  If I fork or
rogue-patch your project, I am saying: "you made a wrong decision by failing to
take the project where I am taking it"; and anyone who uses my forked variation
is endorsing this challenge.  But this in itself would be a fair challenge,
albeit extreme; it's the sharpest end of peer review.  It's therefore not
sufficient in itself to account for the taboos, though it doubtless contributes
force to them.

All three taboo behaviors inflict global harm on the open-source community as
well as local harm on the victim(s).  Implicitly they damage the entire
community by decreasing each potential contributor's perceived likelihood that
gift/productive behavior will be rewarded.

It's important to note that there are alternate candidate explanations for two
of these three taboos.

First, hackers often explain their antipathy to forking projects by bemoaning
the wasteful duplication of work it would imply as the child products evolve on
more-or-less parallel courses into the future.  They may also observe that
forking tends to split the co-developer community, leaving both child projects
with fewer brains to use than the parent.

A respondent has pointed out that it is unusual for more than one offspring of a
fork to survive with significant `market share' into the long term.  This
strengthens the incentives for all parties to cooperate and avoid forking,
because it's hard to know in advance who will be on the losing side and see a
lot of their work either disappear entirely or languish in obscurity.

It has also been pointed out that the simple fact that forks are likely to
produce contention and dispute is enough to motivate social pressure against
them.  Contention and dispute disrupt the teamwork that is necessary for each
individual contributor to reach his or her goals.

Dislike of rogue patches is often explained by the objection that they can
create compatibility problems between the daughter versions, complicate
bug-tracking enormously, and inflict work on maintainers who have quite enough
to do catching their own mistakes.

There is considerable truth to these explanations, and they certainly do their
bit to reinforce the Lockean logic of ownership.  But while intellectually
attractive, they fail to explain why so much emotion and territoriality gets
displayed on the infrequent occasions that the taboos get bent or broken—not
just by the injured parties, but by bystanders and observers who often react
quite harshly.  Cold-blooded concerns about duplication of work and maintainance
hassles simply do not sufficiently explain the observed behavior.

Then, too, there is the third taboo.  It's hard to see how anything but the
reputation-game analysis can explain this.  The fact that this taboo is seldom
analyzed much more deeply than ``It wouldn't be fair'' is revealing in its own
way, as we shall see in the next section.

\section{The Problem of Ego}

At the beginning of this essay I mentioned that the unconscious adaptive
knowledge of a culture is often at odds with its conscious ideology.  We've seen
one major example of this already in the fact that Lockean ownership customs
have been widely followed despite the fact that they violate the stated intent
of the standard licenses.

I have observed another interesting example of this phenomenon when discussing
the reputation-game analysis with hackers.  This is that many hackers resisted
the analysis and showed a strong reluctance to admit that their behavior was
motivated by a desire for peer repute or, as I incautiously labeled it at the
time, `ego satisfaction'.

This illustrates an interesting point about the hacker culture.  It consciously
distrusts and despises egotism and ego-based motivations; self-promotion tends
to be mercilessly criticized, even when the community might appear to have
something to gain from it.  So much so, in fact, that the culture's `big men'
and tribal elders are required to talk softly and humorously deprecate
themselves at every turn in order to maintain their status.  How this attitude
meshes with an incentive structure that apparently runs almost entirely on ego
cries out for explanation.

A large part of it, certainly, stems from the generally negative Europo-American
attitude towards `ego'.  The cultural matrix of most hackers teaches them that
desiring ego satisfaction is a bad (or at least immature) motivation; that ego
is at best an eccentricity tolerable only in prima donnas and often an actual
sign of mental pathology.  Only sublimated and disguised forms like ``peer
repute'', ``self-esteem'', ``professionalism'' or ``pride of accomplishment''
are generally acceptable.

I could write an entire other essay on the unhealthy roots of this part of our
cultural inheritance, and the astonishing amount of self-deceptive harm we do by
believing (against all the evidence of psychology and behavior) that we ever
have truly `selfless' motives.  Perhaps I would, if Friedrich Wilhelm Nietzsche
and Ayn Rand had not already done an entirely competent job (whatever their
other failings) of deconstructing `altruism' into unacknowledged kinds of
self-interest.

But I am not doing moral philosophy or psychology here, so I will simply observe
one minor kind of harm done by the belief that ego is evil, which is this: it
has made it emotionally difficult for many hackers to consciously understand the
social dynamics of their own culture!

But we are not quite done with this line of investigation.  The surrounding
culture's taboo against visibly ego-driven behavior is so much intensified in
the hacker (sub)culture that one must suspect it of having some sort of special
adaptive function for hackers.  Certainly the taboo is weaker (or nonexistent)
among many other gift cultures, such as the peer cultures of theater people or
the very wealthy.

\section{The Value of Humility}

Having established that prestige is central to the hacker culture's reward
mechanisms, we now need to understand why it has seemed so important that this
fact remain semi-covert and largely unadmitted.

The contrast with the pirate culture is instructive.  In that culture,
status-seeking behavior is overt and even blatant.  These crackers seek acclaim
for releasing ``zero-day warez'' (cracked software redistributed on the day of
the original uncracked version's release) but are closemouthed about how they do
it.  These magicians don't like to give away their tricks.  And, as a result,
the knowledge base of the cracker culture as a whole increases only slowly.

In the hacker community, by contrast, one's work is one's statement.  There's a
very strict meritocracy (the best craftsmanship wins) and there's a strong ethos
that quality should (indeed must) be left to speak for itself.  The best brag is
code that ``just works'', and that any competent programmer can see is good
stuff.  Thus, the hacker culture's knowledge base increases rapidly.

The taboo against ego-driven posturing therefore increases productivity.  But
that's a second-order effect; what is being directly protected here is the
quality of the information in the community's peer-evaluation system.  That is,
boasting or self-importance is suppressed because it behaves like noise tending
to corrupt the vital signals from experiments in creative and cooperative
behavior.

For very similar reasons, attacking the author rather than the code is not done.
There is an interesting subtlety here that reinforces the point; hackers feel
very free to flame each other over ideological and personal differences, but it
is unheard of for any hacker to publicly attack another's competence at
technical work (even private criticism is unusual and tends to be muted in
tone).  Bug-hunting and criticism are always project-labeled, not
person-labeled.

Furthermore, past bugs are not automatically held against a developer; the fact
that a bug has been fixed is generally considered more important than the fact
that one used to be there.  As one respondent observed, one can gain status by
fixing `Emacs bugs', but not by fixing `Richard Stallman's bugs'—and it would be
considered extremely bad form to criticize Stallman for old Emacs bugs that have
since been fixed.

This makes an interesting contrast with many parts of academia, in which
trashing putatively defective work by others is an important mode of gaining
reputation.  In the hacker culture, such behavior is rather heavily tabooed—so
heavily, in fact, that the absence of such behavior did not present itself to me
as a datum until that one respondent with an unusual perspective pointed it out
nearly a full year after this essay was first published!

The taboo against attacks on competence (not shared with academia) is even more
revealing than the (shared) taboo on posturing, because we can relate it to a
difference between academia and hackerdom in their communications and support
structures.

The hacker culture's medium of gifting is intangible, its communications
channels are poor at expressing emotional nuance, and face-to-face contact among
its members is the exception rather than the rule.  This gives it a lower
tolerance of noise than most other gift cultures, and goes a long way to explain
both the taboo against posturing and the taboo against attacks on competence.
Any significant incidence of flames over hackers' competence would intolerably
disrupt the culture's reputation scoreboard.

The same vulnerability to noise explains the model of public humility required
of the hacker community's tribal elders.  They must be seen to be free of boast
and posturing so the taboo against dangerous noise will hold.  [DC]

Talking softly is also functional if one aspires to be a maintainer of a
successful project; one must convince the community that one has good judgement,
because most of the maintainer's job is going to be judging other people's code.
Who would be inclined to contribute work to someone who clearly can't judge the
quality of their own code, or whose behavior suggests they will attempt to
unfairly hog the reputation return from the project? Potential contributors want
project leaders with enough humility and class to be able to to say, when
objectively appropriate, ``Yes, that does work better than my version, I'll use
it''—and to give credit where credit is due.

Yet another reason for humble behavior is that in the open source world, you
seldom want to give the impression that a project is `done'.  This might lead a
potential contributor not to feel needed.  The way to maximize your leverage is
to be humble about the state of the program.  If one does one's bragging through
the code, and then says ``Well shucks, it doesn't do x, y, and z, so it can't be
that good'', patches for x, y, and z will often swiftly follow.

Finally, I have personally observed that the self-deprecating behavior of some
leading hackers reflects a real (and not unjustified) fear of becoming the
object of a personality cult.  Linus Torvalds and Larry Wall both provide clear
and numerous examples of such avoidance behavior.  Once, on a dinner expedition
with Larry Wall, I joked ``You're the alpha hacker here—you get to pick the
restaurant''.  He flinched noticeably.  And rightly so; failing to distinguish
their shared values from the personalities of their leaders has ruined a good
many voluntary communities, a pattern of which Larry and Linus cannot fail to be
fully aware.  On the other hand, most hackers would love to have Larry's
problem, if they could but bring themselves to admit it.

\section{Global Implications of the Reputation-Game Model}

The reputation-game analysis has some more implications that may not be
immediately obvious.  Many of these derive from the fact that one gains more
prestige from founding a successful project than from cooperating in an existing
one.  One also gains more from projects that are strikingly innovative, as
opposed to being `me, too' incremental improvements on software that already
exists.  On the other hand, software that nobody but the author understands or
has a need for is a non-starter in the reputation game, and it's often easier to
attract good notice by contributing to an existing project than it is to get
people to notice a new one.  Finally, it's much harder to compete with an
already successful project than it is to fill an empty niche.

Thus, there's an optimum distance from one's neighbors (the most similar
competing projects).  Too close and one's product will be a ``me, too!'' of
limited value, a poor gift (one would be better off contributing to an existing
project).  Too far away, and nobody will be able to use, understand, or perceive
the relevance of one's effort (again, a poor gift).  This creates a pattern of
homesteading in the noosphere that rather resembles that of settlers spreading
into a physical frontier—not random, but like a diffusion-limited fractal.
Projects tend to get started to fill functional gaps near the frontier (see [NO]
for further discussion of the lure of novelty).

Some very successful projects become `category killers'; nobody wants to
homestead anywhere near them because competing against the established base for
the attention of hackers would be too hard.  People who might otherwise found
their own distinct efforts end up, instead, adding extensions for these big,
successful projects.  The classic `category killer' example is GNU Emacs; its
variants fill the ecological niche for a fully-programmable editor so completely
that no competitor has gotten much beyond the one-man project stage since the
early 1980s.  Instead, people write Emacs modes.

Globally, these two tendencies (gap-filling and category-killers) have driven a
broadly predictable trend in project starts over time.  In the 1970s most of the
open source that existed was toys and demos.  In the 1980s the push was in
development and Internet tools.  In the 1990s the action shifted to operating
systems.  In each case, a new and more difficult level of problems was attacked
when the possibilities of the previous one had been nearly exhausted.

This trend has interesting implications for the near future.  In early 1998,
Linux looks very much like a category-killer for the niche `open-source
operating systems'—people who might otherwise write competing operating systems
are now writing Linux device drivers and extensions instead.  And most of the
lower-level tools the culture ever imagined having as open source already exist.
What's left?

Applications.  As the third millenium begins, it seems safe to predict that
open-source development effort will increasingly shift towards the last virgin
territory—programs for non-techies.  A clear early indicator was the development
of GIMP, the Photoshop-like image workshop that is open source's first major
application with the kind of end-user–friendly GUI interface considered de
rigueur in commercial applications for the last decade.  Another is the amount
of buzz surrounding application-toolkit projects like KDE and GNOME.

A respondent to this essay has pointed out that the homesteading analogy also
explains why hackers react with such visceral anger to Microsoft's ``embrace and
extend'' policy of complexifying and then closing up Internet protocols.  The
hacker culture can coexist with most closed software; the existence of Adobe
Photoshop, for example, does not make the territory near GIMP (its open-source
equivalent) significantly less attractive.  But when Microsoft succeeds at
de-commoditizing [HD] a protocol so that only Microsoft's own programmers can
write software for it, they do not merely harm customers by extending their
monopoly; they also reduce the amount and quality of noosphere available for
hackers to homestead and cultivate.  No wonder hackers often refer to
Microsoft's strategy as ``protocol pollution''; they are reacting exactly like
farmers watching someone poison the river they water their crops with!

Finally, the reputation-game analysis explains the oft-cited dictum that you do
not become a hacker by calling yourself a hacker—you become a hacker when other
hackers call you a hacker [KN].  A `hacker', considered in this light, is
somebody who has shown (by contributing gifts) that he or she both has technical
ability and understands how the reputation game works.  This judgement is mostly
one of awareness and acculturation, and can be delivered only by those already
well inside the culture.

\section{How Fine a Gift?}

There are consistent patterns in the way the hacker culture values contributions
and returns peer esteem for them.  It's not hard to observe the following rules:

1.  If it doesn't work as well as I have been led to expect it will, it's no
good—no matter how clever and original it is.

Note the phrase `led to expect'.  This rule is not a demand for perfection; beta
and experimental software is allowed to have bugs.  It's a demand that the user
be able to accurately estimate risks from the stage of the project and the
developers' representations about it.

This rule underlies the fact that open-source software tends to stay in beta for
a long time, and not get even a 1.0 version number until the developers are very
sure it will not hand out a lot of nasty surprises.  In the closed-source world,
Version 1.0 means ``Don't touch this if you're prudent.''; in the open-source
world it reads something more like ``The developers are willing to bet their
reputations on this.''

2.  Work that extends the noosphere is better than work that duplicates an
existing piece of functional territory.

The naive way to put this would have been: Original work is better than mere
duplication of the functions of existing software.  But it's not actually quite
that simple.  Duplicating the functions of existing closed software counts as
highly as original work if by doing so you break open a closed protocol or
format and make that territory newly available.

Thus, for example, one of the highest-prestige projects in the present
open-source world is Samba—the code that allows Unix machines to act as clients
or servers for Microsoft's proprietary SMB file-sharing protocol.  There is very
little creative work to be done here; it's mostly an issue of getting the
reverse-engineered details right.  Nevertheless, the members of the Samba group
are perceived as heroes because they neutralize a Microsoft effort to lock in
whole user populations and cordon off a big section of the noosphere.

3.  Work that makes it into a major distribution is better than work that
doesn't.  Work carried in all major distributions is most prestigious.

The major distributions include not just the big Linux distributions like Red Hat, Debian, Caldera, and SuSE., but other collections that are understood to have reputations of their own to maintain and thus implicitly certify quality —like BSD distributions or the Free Software Foundation source collection.

4.  Utilization is the sincerest form of flattery—and category killers are
better than also-rans.

Trusting the judgment of others is basic to the peer-review process.  It's
necessary because nobody has time to review all possible alternatives.  So work
used by lots of people is considered better than work used by a few,

To have done work so good that nobody cares to use the alternatives any more is
therefore to have earned huge prestige.  The most possible peer esteem comes
from having done widely popular, category-killing original work that is carried
by all major distributions.  People who have pulled this off more than once are
half-seriously referred to as `demigods'.

5.  Continued devotion to hard, boring work (like debugging, or writing
documentation) is more praiseworthy than cherrypicking the fun and easy hacks.

This norm is how the community rewards necessary tasks that hackers would not
naturally incline towards.  It is to some extent contradicted by:

6.  Nontrivial extensions of function are better than low-level patches and
debugging.

The way this seems to work is that on a one-shot basis, adding a feature is
likely to get more reward than fixing a bug—unless the bug is exceptionally
nasty or obscure, such that nailing it is itself a demonstration of unusual
skill and cleverness.  But when these behaviors are extended over time, a person
with a long history of paying attention to and nailing even ordinary bugs may
well out-rank someone who has spent a similar amount of effort adding easy
features.

A respondent has pointed out that these rules interact in interesting ways and
do not necessarily reward highest possible utility all the time.  Ask a hacker
whether he's likely to become better known for a brand new tool of his own or
for extensions to someone else's and the answer ``new tool'' will not be in
doubt.  But ask about (a) a brand new tool which is only used a few times a day
invisibly by the OS but which rapidly becomes a category killer, versus (b)
several extensions to an existing tool which are neither especially novel nor
category-killers, but are daily used and daily visible to a huge number of users

and you are likely to get some hesitation before the hacker settles on (a).
These alternatives are about evenly stacked.

Said respondent gave this question point for me by adding ``Case (a) is
fetchmail; case (b) is your many Emacs extensions, like vc.el and gud.el.'' And
indeed he is correct; I am more likely to be tagged ``the author of fetchmail''
than ``author of a boatload of Emacs modes'', even though the latter probably
have had higher total utility over time.

What may be going on here is simply that work with a novel `brand identity' gets
more notice than work aggregated to an existing `brand'.  Elucidation of these
rules, and what they tell us about the hacker culture's scoreboarding system,
would make a good topic for further investigation.

\section{Noospheric Property and the Ethology of Territory}

To understand the causes and consequences of Lockean property customs, it will
help us to look at them from yet another angle; that of animal ethology,
specifically the ethology of territory.

Property is an abstraction of animal territoriality, which evolved as a way of
reducing intraspecies violence.  By marking his bounds, and respecting the
bounds of others, a wolf diminishes his chances of being in a fight that could
weaken or kill him and make him less reproductively successful.  Similarly, the
function of property in human societies is to prevent inter-human conflict by
setting bounds that clearly separate peaceful behavior from aggression.

It is fashionable in some circles to describe human property as an arbitrary
social convention, but this is dead wrong.  Anybody who has ever owned a dog who
barked when strangers came near its owner's property has experienced the
essential continuity between animal territoriality and human property.  Our
domesticated cousins of the wolf know, instinctively, that property is no mere
social convention or game, but a critically important evolved mechanism for the
avoidance of violence.  (This makes them smarter than a good many human
political theorists.)

Claiming property (like marking territory) is a performative act, a way of
declaring what boundaries will be defended.  Community support of property
claims is a way to minimize friction and maximize cooperative behavior.  These
things remain true even when the ``property claim'' is much more abstract than a
fence or a dog's bark, even when it's just the statement of the project
maintainer's name in a README file.  It's still an abstraction of
territoriality, and (like other forms of property) based in territorial
instincts evolved to assist conflict resolution.

This ethological analysis may at first seem very abstract and difficult to
relate to actual hacker behavior.  But it has some important consequences.  One
is in explaining the popularity of World Wide Web sites, and especially why
open-source projects with websites seem so much more `real' and substantial than
those without them.

Considered objectively, this seems hard to explain.  Compared to the effort
involved in originating and maintaining even a small program, a web page is
easy, so it's hard to consider a web page evidence of substance or unusual
effort.

Nor are the functional characteristics of the Web itself sufficient explanation.
The communication functions of a web page can be as well or better served by a
combination of an FTP site, a mailing list, and Usenet postings.  In fact it's
quite unusual for a project's routine communications to be done over the Web
rather than via a mailing list or newsgroup.  Why, then, the popularity of
websites as project homes?

The metaphor implicit in the term `home page' provides an important clue.  While
founding an open-source project is a territorial claim in the noosphere (and
customarily recognized as such) it is not a terribly compelling one on the
psychological level.  Software, after all, has no natural location and is
instantly reduplicable.  It's assimilable to our instinctive notions of
`territory' and `property', but only after some effort.

A project home page concretizes an abstract homesteading in the space of
possible programs by expressing it as `home' territory in the more
spatially-organized realm of the World Wide Web.  Descending from the noosphere
to `cyberspace' doesn't get us all the way to the real world of fences and
barking dogs yet, but it does hook the abstract property claim more securely to
our instinctive wiring about territory.  And this is why projects with web pages
seem more `real'.

This point is much strengthened by hyperlinks and the existence of good search
engines.  A project with a web page is much more likely to be noticed by
somebody exploring its neighborhood in the noosphere; others will link to it,
searches will find it.  A web page is therefore a better advertisement, a more
effective performative act, a stronger claim on territory.

This ethological analysis also encourages us to look more closely at mechanisms
for handling conflict in the open-source culture.  It leads us to expect that,
in addition to maximizing reputation incentives, ownership customs should also
have a role in preventing and resolving conflicts.

\section{Causes of Conflict}

In conflicts over open-source software we can identify four major issues:
\begin{itemize}
\item Who gets to make binding decisions about a project?
\item Who gets credit or blame for what?
\item How to reduce duplication of effort and prevent rogue versions from
  complicating bug tracking?
\item What is the Right Thing, technically speaking?
\end{itemize}
If we take a second look at the ``What is the Right Thing'' issue, however, it
tends to vanish.  For any such question, either there is an objective way to
decide it accepted by all parties or there isn't.  If there is, game over and
everybody wins.  If there isn't, it reduces to ``Who decides?''.

Accordingly, the three problems a conflict-resolution theory has to resolve
about a project are (a) where the buck stops on design decisions, (b) how to
decide which contributors are credited and how, and (c) how to keep a project
group and product from fissioning into multiple branches.

The role of ownership customs in resolving issues (a) and (c) is clear.  Custom
affirms that the owners of the project make the binding decisions.  We have
previously observed that custom also exerts heavy pressure against dilution of
ownership by forking.

It's instructive to notice that these customs make sense even if one forgets the
reputation game and examines them from within a pure `craftmanship' model of the
hacker culture.  In this view these customs have less to do with the dilution of
reputation incentives than with protecting a craftsman's right to execute his
vision in his chosen way.

The craftsmanship model is not, however, sufficient to explain hacker customs
about issue (b), who gets credit for what—because a pure craftsman, one
unconcerned with the reputation game, would have no motive to care.  To analyze
these, we need to take the Lockean theory one step further and examine conflicts
and the operation of property rights within projects as well as between them.

\section{Project Structures and Ownership}


The trivial case is that in which the project has a single owner/maintainer.  In
that case there is no possible conflict.  The owner makes all decisions and
collects all credit and blame.  The only possible conflicts are over succession
issues—who gets to be the new owner if the old one disappears or loses interest.
The community also has an interest, under issue (c), in preventing forking.
These interests are expressed by a cultural norm that an owner/maintainer should
publicly hand title to someone if he or she can no longer maintain the project.

The simplest non-trivial case is when a project has multiple co-maintainers
working under a single ``benevolent dictator'' who owns the project.  Custom
favors this mode for group projects; it has been shown to work on projects as
large as the Linux kernel or Emacs, and solves the ``who decides'' problem in a
way that is not obviously worse than any of the alternatives.

Typically, a benevolent-dictator organization evolves from an owner-maintainer
organization as the founder attracts contributors.  Even if the owner stays
dictator, it introduces a new level of possible disputes over who gets credited
for what parts of the project.

In this situation, custom places an obligation on the owner/dictator to credit
contributors fairly (through, for example, appropriate mentions in README or
history files).  In terms of the Lockean property model, this means that by
contributing to a project you earn part of its reputation return (positive or
negative).

Pursuing this logic, we see that a ``benevolent dictator'' does not in fact own
his entire project absolutely.  Though he has the right to make binding
decisions, he in effect trades away shares of the total reputation return in
exchange for others' work.  The analogy with sharecropping on a farm is almost
irresistible, except that a contributor's name stays in the credits and
continues to ``earn'' to some degree even after that contributor is no longer
active.

As benevolent-dictator projects add more participants, they tend to develop two
tiers of contributors; ordinary contributors and co-developers.  A typical path
to becoming a co-developer is taking responsibility for a major subsystem of the
project.  Another is to take the role of ``lord high fixer'', characterizing and
fixing many bugs.  In this way or others, co-developers are the contributors who
make a substantial and continuing investment of time in the project.

The subsystem-owner role is particularly important for our analysis and deserves
further examination.  Hackers like to say that ``authority follows
responsibility''.  A co-developer who accepts maintainance responsibility for a
given subsystem generally gets to control both the implementation of that
subsystem and its interfaces with the rest of the project, subject only to
correction by the project leader (acting as architect).  We observe that this
rule effectively creates enclosed properties on the Lockean model within a
project, and has exactly the same conflict-prevention role as other property
boundaries.

By custom, the ``dictator'' or project leader in a project with co-developers is
expected to consult with those co-developers on key decisions.  This is
especially so if the decision concerns a subsystem that a co-developer ``owns''
(that is, has invested time in and taken responsibility for).  A wise leader,
recognizing the function of the project's internal property boundaries, will not
lightly interfere with or reverse decisions made by subsystem owners.

Some very large projects discard the ``benevolent dictator'' model entirely.
One way to do this is turn the co-developers into a voting committee (as with
Apache).  Another is rotating dictatorship, in which control is occasionally
passed from one member to another within a circle of senior co-developers; the
Perl developers organize themselves this way.

Such complicated arrangements are widely considered unstable and difficult.
Clearly this perceived difficulty is largely a function of the known hazards of
design-by-committee, and of committees themselves; these are problems the hacker
culture consciously understands.  However, I think some of the visceral
discomfort hackers feel about committee or rotating-chair organizations is that
they're hard to fit into the unconscious Lockean model hackers use for reasoning
about the simpler cases.  It's problematic, in these complex organizations, to
do an accounting of either ownership in the sense of control or ownership of
reputation returns.  It's hard to see where the internal boundaries are, and
thus hard to avoid conflict unless the group enjoys an exceptionally high level
of harmony and trust.

\section{Conflict and Conflict Resolution}

We've seen that within projects, an increasing complexity of roles is expressed
by a distribution of design authority and partial property rights.  While this
is an efficient way to distribute incentives, it also dilutes the authority of
the project leader—most importantly, it dilutes the leader's authority to squash
potential conflicts.

While technical arguments over design might seem the most obvious risk for
internecine conflict, they are seldom a serious cause of strife.  These are
usually relatively easily resolved by the territorial rule that authority
follows responsibility.

Another way of resolving conflicts is by seniority—if two contributors or groups
of contributors have a dispute, and the dispute cannot be resolved objectively,
and neither owns the territory of the dispute, the side that has put the most
work into the project as a whole (that is, the side with the most property
rights in the whole project) wins.

(Equivalently, the side with the least invested loses.  Interestingly this
happens to be the same heuristic that many relational database engines use to
resolve deadlocks.  When two threads are deadlocked over resources, the side
with the least invested in the current transaction is selected as the deadlock
victim and is terminated.  This usually selects the longest running transaction,
or the more senior, as the victor.)

These rules generally suffice to resolve most project disputes.  When they do
not, fiat of the project leader usually suffices.  Disputes that survive both
these filters are rare.

Conflicts do not, as a rule, become serious unless these two criteria
("authority follows responsibility" and "seniority wins") point in different
directions, and the authority of the project leader is weak or absent.  The most
obvious case in which this may occur is a succession dispute following the
disappearance of the project lead.  I have been in one fight of this kind.  It
was ugly, painful, protracted, only resolved when all parties became exhausted
enough to hand control to an outside person, and I devoutly hope I am never
anywhere near anything of the kind again.

Ultimately, all of these conflict-resolution mechanisms rest on the entire
hacker community's willingness to enforce them.  The only available enforcement
mechanisms are flaming and shunning—public condemnation of those who break
custom, and refusal to cooperate with them after they have done so.

\section{Acculturation Mechanisms and the Link to Academia}

An early version of this essay posed the following research question: how does
the community inform and instruct its members as to its customs? Are the customs
self-evident or self-organizing at a semi-conscious level? Are they taught by
example? Are they taught by explicit instruction?

Teaching by explicit instruction is clearly rare, if only because few explicit
descriptions of the culture's norms have existed for instructional use up to
now.

Many norms are taught by example.  To cite one very simple case, there is a norm
that every software distribution should have a file called README or READ.ME
that contains first-look instructions for browsing the distribution.  This
convention has been well established since at least the early 1980s; it has
even, occasionally, been written down.  But one normally derives it from looking
at many distributions.

On the other hand, some hacker customs are self-organizing once one has acquired
a basic (perhaps unconscious) understanding of the reputation game.  Most
hackers never have to be taught the three taboos I listed earlier in this essay,
or at least would claim if asked that they are self-evident rather than
transmitted.  This phenomenon invites closer analysis—and perhaps we can find
its explanation in the process by which hackers acquire knowledge about the
culture.

Many cultures use hidden clues (more precisely `mysteries' in the
religio/mystical sense) as an acculturation mechanism.  These are secrets that
are not revealed to outsiders, but are expected to be discovered or deduced by
the aspiring newbie.  To be accepted inside, one must demonstrate that one both
understands the mystery and has learned it in a culturally sanctioned way.

The hacker culture makes unusually conscious and extensive use of such clues or
tests.  We can see this process operating at at least three levels:
\begin{itemize}
\item Password-like specific mysteries.  As one example, there is a Usenet
  newsgroup called alt.sysadmin.recovery that has a very explicit such secret;
  you cannot post without knowing it, and knowing it is considered evidence you
  are fit to post.  The regulars have a strong taboo against revealing this
  secret.
\item The requirement of initiation into certain technical mysteries.  One must
  absorb a good deal of technical knowledge before one can give valued gifts
  (e.g.\ one must know at least one of the major computer languages).  This
  requirement functions in the large in the way hidden clues do in the small, as
  a filter for qualities (such as capability for abstract thinking, persistence,
  and mental flexibility) that are necessary to function in the culture.
\item Social-context mysteries.  One becomes involved in the culture through
  attaching oneself to specific projects.  Each project is a live social context
  of hackers that the would-be contributor has to investigate and understand
  socially as well as technically in order to function.  (Concretely, a common
  way one does this is by reading the project's web pages and/or email
  archives.) It is through these project groups that newbies experience the
  behavioral example of experienced hackers.
\end{itemize}

In the process of acquiring these mysteries, the would-be hacker picks up
contextual knowledge that (after a while) does make the three taboos and other
customs seem `self-evident'.

One might, incidentally, argue that the structure of the hacker gift culture
itself is its own central mystery.  One is not considered acculturated
(concretely: no one will call you a hacker) until one demonstrates a gut-level
understanding of the reputation game and its implied customs, taboos, and
usages.  But this is trivial; all cultures demand such understanding from
would-be joiners.  Furthermore the hacker culture evinces no desire to have its
internal logic and folkways kept secret—or, at least, nobody has ever flamed me
for revealing them!

Respondents to this essay too numerous to list have pointed out that hacker
ownership customs seem intimately related to (and may derive directly from) the
practices of the academic world, especially the scientific research commmunity.
This research community has similar problems in mining a territory of
potentially productive ideas, and exhibits very similar adaptive solutions to
those problems in the ways it uses peer review and reputation.

Since many hackers have had formative exposure to academia (it's common to learn
how to hack while in college) the extent to which academia shares adaptive
patterns with the hacker culture is of more than casual interest in
understanding how these customs are applied.

Obvious parallels with the hacker `gift culture' as I have characterized it
abound in academia.  Once a researcher achieves tenure, there is no need to
worry about survival issues.  (Indeed, the concept of tenure can probably be
traced back to an earlier gift culture in which ``natural philosophers'' were
primarily wealthy gentlemen with time on their hands to devote to research.) In
the absence of survival issues, reputation enhancement becomes the driving goal,
which encourages sharing of new ideas and research through journals and other
media.  This makes objective functional sense because scientific research, like
the hacker culture, relies heavily on the idea of `standing upon the shoulders
of giants', and not having to rediscover basic principles over and over again.

Some have gone so far as to suggest that hacker customs are merely a reflection
of the research community's folkways and have actually (in most cases) been
acquired there by individual hackers.  This probably overstates the case, if
only because hacker custom seems to be readily acquired by intelligent
high-schoolers!

\section{Gift Outcompetes Exchange}

There is a more interesting possibility here.  I suspect academia and the hacker
culture share adaptive patterns not because they're genetically related, but
because they've both evolved the one most optimal social organization for what
they're trying to do, given the laws of nature and the instinctive wiring of
human beings.  The verdict of history seems to be that free-market capitalism is
the globally optimal way to cooperate for economic efficiency; perhaps, in a
similar way, the reputation-game gift culture is the globally optimal way to
cooperate for generating (and checking!) high-quality creative work.

Support for this theory becomes from a large body of psychological studies on
the interaction between art and reward [GNU].  These studies have received less
attention than they should, in part perhaps because their popularizers have
shown a tendency to overinterpret them into general attacks against the free
market and intellectual property.  Nevertheless, their results do suggest that
some kinds of scarcity-economics rewards actually decrease the productivity of
creative workers such as programmers.

Psychologist Theresa Amabile of Brandeis University, cautiously summarizing the
results of a 1984 study of motivation and reward, observed ``It may be that
commissioned work will, in general, be less creative than work that is done out
of pure interest.''.  Amabile goes on to observe that ``The more complex the
activity, the more it's hurt by extrinsic reward.'' Interestingly, the studies
suggest that flat salaries don't demotivate, but piecework rates and bonuses do.

Thus, it may be economically smart to give performance bonuses to people who
flip burgers or dug ditches, but it's probably smarter to decouple salary from
performance in a programming shop and let people choose their own projects (both
trends that the open-source world takes to their logical conclusions).  Indeed,
these results suggest that the only time it is a good idea to reward performance
in programming is when the programmer is so motivated that he or she would have
worked without the reward!

Other researchers in the field are willing to point a finger straight at the
issues of autonomy and creative control that so preoccupy hackers.  ``To the
extent one's experience of being self-determined is limited,'' said Richard
Ryan, associate psychology professor at the University of Rochester, ``one's
creativity will be reduced as well.''

In general, presenting any task as a means rather than an end in itself seems to
demotivate.  Even winning a competition with others or gaining peer esteem can
be demotivating in this way if the victory is experienced as work for reward
(which may explain why hackers are culturally prohibited from explicitly seeking
or claiming that esteem).

To complicate the management problem further, controlling verbal feedback seems
to be just as demotivating as piecework payment.  Ryan found that corporate
employees who were told, ``Good, you're doing as you should'' were
``significantly less intrinsically motivated than those who received feedback
informationally.''

It may still be intelligent to offer incentives, but they have to come without
attachments to avoid gumming up the works.  There is a critical difference (Ryan
observes) between saying, ``I'm giving you this reward because I recognize the
value of your work'', and ``You're getting this reward because you've lived up
to my standards.'' The first does not demotivate; the second does.

In these psychological observations we can ground a case that an open-source
development group will be substantially more productive (especially over the
long term, in which creativity becomes more critical as a productivity
multiplier) than an equivalently sized and skilled group of closed-source
programmers (de)motivated by scarcity rewards.

This suggests from a slightly different angle one of the speculations in The
Cathedral And The Bazaar; that, ultimately, the industrial/factory mode of
software production was doomed to be outcompeted from the moment capitalism
began to create enough of a wealth surplus that many programmers could live in a
post-scarcity gift culture.

Indeed, it seems the prescription for highest software productivity is almost a
Zen paradox; if you want the most efficient production, you must give up trying
to make programmers produce.  Handle their subsistence, give them their heads,
and forget about deadlines.  To a conventional manager this sounds crazily
indulgent and doomed—but it is exactly the recipe with which the open-source
culture is now clobbering its competition.

\section{Conclusion: From Custom to Customary Law}

We have examined the customs which regulate the ownership and control of
open-source software.  We have seen how they imply an underlying theory of
property rights homologous to the Lockean theory of land tenure.  We have
related that to an analysis of the hacker culture as a `gift culture' in which
participants compete for prestige by giving time, energy, and creativity away.
We have examined the implications of this analysis for conflict resolution in
the culture.

The next logical question to ask is "Why does this matter?" Hackers developed
these customs without conscious analysis and (up to now) have followed them
without conscious analysis.  It's not immediately clear that conscious analysis
has gained us anything practical—unless, perhaps, we can move from description
to prescription and deduce ways to improve the functioning of these customs.

We have found a close logical analogy for hacker customs in the theory of land
tenure under the Anglo-American common-law tradition.  Historically [Miller],
the European tribal cultures that invented this tradition improved their
dispute-resolution systems by moving from a system of unarticulated,
semi-conscious custom to a body of explicit customary law memorized by tribal
wisemen—and eventually, written down.

Perhaps, as our population rises and acculturation of all new members becomes
more difficult, it is time for the hacker culture to do something analogous—to
develop written codes of good practice for resolving the various sorts of
disputes that can arise in connection with open-source projects, and a tradition
of arbitration in which senior members of the community may be asked to mediate
disputes.

The analysis in this essay suggests the outlines of what such a code might look
like, making explicit that which was previously implicit.  No such codes could
be imposed from above; they would have to be voluntarily adopted by the founders
or owners of individual projects.  Nor could they be completely rigid, as the
pressures on the culture are likely to change over time.  Finally, for
enforcement of such codes to work, they would have to reflect a broad consensus
of the hacker tribe.

I have begun work on such a code, tentatively titled the "Malvern Protocol"
after the little town where I live.  If the general analysis in this paper
becomes sufficiently widely accepted, I will make the Malvern Protocol publicly
available as a model code for dispute resolution.  Parties interested in
critiquing and developing this code, or just offering feedback on whether they
think it's a good idea or not, are invited to contact me by email.

\section{Questions for Further Research}

The culture's (and my own) understanding of large projects that don't follow a
benevolent-dictator model is weak.  Most such projects fail.  A few become
spectacularly successful and important (Perl, Apache, KDE).  Nobody really
understands where the difference lies.  There's a vague sense abroad that each
such project is sui generis and stands or falls on the group dynamic of its
particular members, but is this true or are there replicable strategies that a
group can follow?

\section{Notes}

Thyrsus Enterprises

<esr@thyrsus.com>

This is version 3.0

Copyright © 2000 Eric S. Raymond

Copyright

Permission is granted to copy, distribute and/or modify this document under the
terms of the Open Publication License, version 2.0.

\$Date: 2002/08/02 09:02:15 \$

\include{bibliography}
\include{acknowledgements}



\end{document}
